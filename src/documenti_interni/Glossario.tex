% template presente
\documentclass[a4paper,12pt]{article}

\usepackage[utf8]{inputenc}
\usepackage[italian]{babel}
\usepackage{graphicx}
\graphicspath{{resources/}{../resources/}{../../resources/}{../../../resources/}{../../../../resources/}}
\usepackage{xcolor}
\usepackage{geometry}
\usepackage{setspace}
\usepackage{colortbl}
\usepackage{hyperref} % link cliccabili
\usepackage{fancyhdr} % intestazioni personalizzate
\usepackage{titlesec} % per formattare le sezioni
\geometry{margin=2.5cm}

\setlength{\parindent}{0pt}
\setstretch{1.2}

% ===== Stile intestazione =====
\pagestyle{fancy}
\fancyhf{}
\fancyhead[L]{\textcolor{gray}{Verbale Riunione - Gruppo 17}}
\fancyfoot[C]{\thepage}

% ===== Formattazione sezioni =====
\titleformat{\section}{\Large\bfseries}{\thesection.}{0.5em}{}
\titleformat{\subsection}{\large\bfseries}{\thesubsection.}{0.5em}{}

\begin{document}

% ======= HEADER UNIVERSITÀ E GRUPPO CENTRATI VERTICALMENTE =======
\vspace*{\fill} % --- Spinge verso il basso l'inizio del contenuto

\begin{center}
    \begin{minipage}{0.25\textwidth}
        \centering
        \includegraphics[width=\linewidth]{logoUni.png}
    \end{minipage}
    \hfill
    \begin{minipage}{0.7\textwidth}
        \raggedright
        {\color{red}\LARGE \textbf{Università degli Studi di Padova}}\\[0.3cm]
        {\large
        Laurea: Informatica\\
        Corso: Ingegneria del Software\\
        Anno Accademico: 2025/26
        }
    \end{minipage}
\end{center}

\vspace{1cm}

\begin{center}
    \begin{minipage}{0.25\textwidth}
        \centering
        \includegraphics[width=\linewidth]{logo.png}
    \end{minipage}
    \hfill
    \begin{minipage}{0.7\textwidth}
        \raggedright
        {\LARGE \textbf{Gruppo 17}}\\[0.3cm]
        {\large
        Nome: BitByBit\\
        Email: swe.bitbybit@gmail.com
        }
    \end{minipage}
\end{center}

\vspace{1.5cm}

\begin{center}
    {\LARGE \textbf{Glossario}}
\end{center}

\vspace*{\fill} % --- Spinge verso l’alto la fine del contenuto, centrando tutto il blocco

\clearpage
% ======= INFO GENERALI =======

% ======= INDICE SU PAGINA DEDICATA =======
\clearpage
\tableofcontents
\thispagestyle{empty} % senza numero di pagina per l'indice
\clearpage

% ---------- REDAZIONE E REVISIONE ----------
\clearpage
\section{Glossario}

\subsection*{Acronimi}
\begin{tabular}{p{0.22\linewidth} p{0.72\linewidth}}
\textbf{API} & Application Programming Interface. \\
\textbf{PoC} & Proof of Concept \\
\textbf{MVP} & Minimum Viable Product. \\
\textbf{RTB} & Requirements and Technology Baseline \\
\textbf{PB} & Product Baseline \\
\textbf{DB} & Database. \\
\textbf{JSON} & JavaScript Object Notation. \\

\end{tabular}

\vspace{0.6cm}

\subsection*{Termini principali}
\begin{description}
  \item[Requisito] Condizione o capacità che il sistema deve soddisfare.
  \item[Requisito funzionale] Funzionalità che il sistema deve offrire.
  \item[Requisito non funzionale] Vincolo di qualità (prestazioni, sicurezza, usabilità, manutentibilità).
  \item[Backlog] Elenco prioritario delle funzionalità, bug e attività da svolgere.
  \item[Sprint] Iterazione di lavoro con obiettivi definiti (2 settimane).
  \item[PoC] prototipo o esperimento usato per dimostrare la fattibilità di una soluzione.
  \item[Repository/Repo] Archivio centralizzato dove viene conservato e gestito il codice sorgente del progetto.
  \item[MVP] Versione minima del prodotto che permette di validare l'ipotesi principale.
  \item[RTB] Insieme dei requisiti approvati e della loro tracciabilità.
  \item[PB] Versione approvata del prodotto pronta al rilascio o alla validazione.
  \item[Prototipo] Versione preliminare per testare design o comportamenti prima dello sviluppo completo.
  \item[Architettura] Organizzazione generale del sistema e dei suoi componenti.
  \item[Componente] Unità software con responsabilità ben definite e interfacce pubbliche.
  \item[API] Interfaccia che consente la comunicazione tra componenti o servizi.
  \item[Database] Sistema di memorizzazione dei dati e la sua struttura definita.
  \item[Normalizzazione] Tecnica per ridurre ridondanza e incongruenze nei dati.
  \item[JSON] Formato per lo scambio di dati tra client e server.
  \item[Test unitario] Verifica automatizzata di singole unità di codice.
  \item[Test di integrazione] Verifica delle interazioni tra più componenti o servizi.
  \item[Repository / Branch] Archivio del codice sorgente e linee di sviluppo parallele.
  \item[Pull Request / Merge Request] Procedura per proporre e rivedere modifiche prima dell'integrazione.
  \item[Issue] Problema da risolvere.
  \item[Task] Attività da svolgere.
  \item[Bug] Difetto da controllare.
  \item[Deployment] Processo di pubblicazione del software in un ambiente esecutivo.
  \item[Licenza] Condizioni legali che regolano l'uso e la distribuzione del software.
\end{description}


\section{Redazione e revisioni del documento}

\begin{center}
\small
\renewcommand{\arraystretch}{1.2} 
\arrayrulecolor{black}
\begin{tabular}{|p{0.1\linewidth}|p{0.18\linewidth}|p{0.22\linewidth}|p{0.20\linewidth}|p{0.22\linewidth}|}
\hline
\rowcolor{gray!60} 
\textbf{Versione} & \textbf{Ruolo} & \textbf{Nome} & \textbf{Data e ora} & \textbf{Descrizione} \\
\hline
\rowcolor{white}
0.1 & Redatto da & Visentin Giovanni & 2025-11-04 15:55 & Stesura iniziale del verbale \\
\hline
\rowcolor{gray!20}
 & Revisione &  &  & Controllo approfondito del verbale \\
\hline
\rowcolor{white}
 & Conferma & Tutti i membri &  & Conferma da parte di tutti del verbale \\
\hline
\rowcolor{gray!20}
 & Modifiche &  &  &  \\
\hline
\end{tabular}
\end{center}

\end{document}
