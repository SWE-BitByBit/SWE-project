% template presente
\documentclass[a4paper,12pt]{article}

\usepackage[utf8]{inputenc}
\usepackage[italian]{babel}
\usepackage{graphicx}
\graphicspath{{resources/}{../resources/}{../../resources/}{../../../resources/}{../../../../resources/}}
\usepackage{xcolor}
\usepackage{geometry}
\usepackage{setspace}
\usepackage{colortbl}
\usepackage{hyperref} % link cliccabili
\hypersetup{
    colorlinks=true,
    linkcolor=blue,
    filecolor=blue,
    urlcolor=blue,     
}
\usepackage{fancyhdr} % intestazioni personalizzate
\usepackage{titlesec} % per formattare le sezioni
\usepackage{float} % per la tabella del glossario
\usepackage{longtable} % per fare in modo che la tabella sia più lunga di una pagina
\geometry{margin=2.5cm}

\setlength{\parindent}{0pt}
\setstretch{1.2}

% ===== Stile intestazione =====
\pagestyle{fancy}
\fancyhf{}
\fancyhead[L]{\textcolor{gray}{Verbale Riunione - Gruppo 17}}
\fancyfoot[C]{\thepage}

% ===== Formattazione sezioni =====
\titleformat{\section}{\Large\bfseries}{\thesection.}{0.5em}{}
\titleformat{\subsection}{\large\bfseries}{\thesubsection.}{0.5em}{}

\begin{document}

% ======= HEADER UNIVERSITÀ E GRUPPO CENTRATI VERTICALMENTE =======
\vspace*{\fill} % --- Spinge verso il basso l'inizio del contenuto

\begin{center}
    \begin{minipage}{0.25\textwidth}
        \centering
        \includegraphics[width=\linewidth]{logoUni.png}
    \end{minipage}
    \hfill
    \begin{minipage}{0.7\textwidth}
        \raggedright
        {\color{red}\LARGE \textbf{Università degli Studi di Padova}}\\[0.3cm]
        {\large
        Laurea: Informatica\\
        Corso: Ingegneria del Software\\
        Anno Accademico: 2025/26
        }
    \end{minipage}
\end{center}

\vspace{1cm}

\begin{center}
    \begin{minipage}{0.25\textwidth}
        \centering
        \includegraphics[width=\linewidth]{logo.png}
    \end{minipage}
    \hfill
    \begin{minipage}{0.7\textwidth}
        \raggedright
        {\LARGE \textbf{Gruppo 17}}\\[0.3cm]
        {\large
        Nome: BitByBit\\
        Email: swe.bitbybit@gmail.com
        }
    \end{minipage}
\end{center}

\vspace{1.5cm}

\begin{center}
    {\LARGE \textbf{Glossario}}
\end{center}

\vspace*{\fill} % --- Spinge verso l’alto la fine del contenuto, centrando tutto il blocco

\clearpage
% ======= INFO GENERALI =======

% ======= INDICE SU PAGINA DEDICATA =======
\clearpage
\tableofcontents
\thispagestyle{empty} % senza numero di pagina per l'indice
\clearpage

% ---------- REDAZIONE E REVISIONE ----------
\clearpage
\section{Glossario}

\subsection*{Acronimi}
\begin{tabular}{p{0.22\linewidth} p{0.72\linewidth}}
\textbf{API} & Application Programming Interface. \\
\textbf{PoC} & Proof of Concept \\
\textbf{MVP} & Minimum Viable Product. \\
\textbf{RTB} & Requirements and Technology Baseline \\
\textbf{PB} & Product Baseline \\
\textbf{DB} & Database. \\
\textbf{JSON} & JavaScript Object Notation. \\

\end{tabular}

\vspace{0.6cm}

\subsection*{Termini principali}

\renewcommand{\arraystretch}{1.2}
\begin{longtable}{p{4cm} p{11cm}}
\textbf{Termine} & \textbf{Definizione} \\ \hline
\endfirsthead

\textbf{Termine} & \textbf{Definizione} \\ \hline
\endhead

Requisito & Condizione o capacità che il sistema deve soddisfare. \\
Requisito funzionale & Funzionalità che il sistema deve offrire. \\
Requisito non funzionale & Vincolo di qualità (prestazioni, sicurezza, usabilità, manutentibilità). \\
Backlog & Elenco prioritario delle funzionalità, bug e attività da svolgere. \\
Sprint & Iterazione di lavoro con obiettivi definiti (2 settimane). \\
PoC & Prototipo o esperimento usato per dimostrare la fattibilità di una soluzione. \\
Repository / Repo & Archivio centralizzato dove viene conservato e gestito il codice sorgente del progetto. \\
MVP & Versione minima del prodotto che permette di validare l'ipotesi principale. \\
RTB & Insieme dei requisiti approvati e della loro tracciabilità. \\
PB & Versione approvata del prodotto pronta al rilascio o alla validazione. \\
Prototipo & Versione preliminare per testare design o comportamenti prima dello sviluppo completo. \\
Architettura & Organizzazione generale del sistema e dei suoi componenti. \\
Componente & Unità software con responsabilità ben definite e interfacce pubbliche. \\
API & Interfaccia che consente la comunicazione tra componenti o servizi. \\
Database & Sistema di memorizzazione dei dati e la sua struttura definita. \\
Normalizzazione & Tecnica per ridurre ridondanza e incongruenze nei dati. \\
JSON & Formato per lo scambio di dati tra client e server. \\
Test unitario & Verifica automatizzata di singole unità di codice. \\
Test di integrazione & Verifica delle interazioni tra più componenti o servizi. \\
Repository / Branch & Archivio del codice sorgente e linee di sviluppo parallele. \\
Pull Request / Merge Request & Procedura per proporre e rivedere modifiche prima dell'integrazione. \\
Issue & Problema da risolvere. \\
Task & Attività da svolgere. \\
Bug & Difetto da controllare. \\
Deployment & Processo di pubblicazione del software in un ambiente esecutivo. \\
Licenza & Condizioni legali che regolano l'uso e la distribuzione del software. \\
Glossario & Elenco termini utilizzati dal gruppo. \\
Verbale & Documento riassuntivo di una riunione. Ciascun Verbale deve specificare Ordine del Giorno, Discussioni fatte, Decisioni intraprese, Issue concrete da svolgere e deve essere verificato da un membro del gruppo diverso da quello che lo ha scritto. \\
To Do & Sezione di un verbale o di un documento di altro tipo in cui si elencano le azioni concrete che ogni membro del gruppo deve svolgere a partire da quanto è stato redatto il documento.

\end{longtable}
\newpage
\section{Nomenclatura}
In questa sezione viene specificato in maniera esatta in che modo vanno nominati file, cartelle e altri artefatti.
\subsection{Date}
Il formato delle date è \textbf{AAAA-MM-GG}. Un esempio di data corretta è 2025-11-04. Ogni data deve  sempre essere costituita da 10 caratteri. Non è consentito abbreviare l'anno (es: 25 al posto di 2025) né scrivere singole cifre senza uno zero prima (es: 5 al posto di 05) né usare lo slash (/) al posto del trattino.

\subsection{Ore}
Ci sono due formati per le ore:
\begin{itemize}
    \item \textbf{Durata}: Per espriemere una durata usare il formato \textbf{Hh Mm} (es: 2h 15m)
    \item \textbf{Ora di orologio}: Per indicare che un evento è avvenuto in una specifica ora, usare il formato \textbf{HH:MM} (es: 15:45)
\end{itemize}

\subsection{Verbali}
Ogni verbale va nominato in relazione alla data a cui si riferisce. Il nome di ogni verbale deve iniziare con "verbale\_" (lettere minuscole), seguito dalla data nel formato concordato. Un esempio di nominazione corretta è "verbale\_2025-11-05". L'unica eccezione è nel caso in cui usare questo formato generi due verbali con lo stesso nome all'interno della stessa cartella. In tal caso è necessario aggiungere, dopo la data, una breve descrizione del contesto relativo al verbale, sostituendo agli spazi il simbolo \_. Ad esempio "verbale\_2025-10-28\_incontro\_c8"

\clearpage
\section{Redazione e revisioni del documento}

\begin{center}
\small
\renewcommand{\arraystretch}{1.2} 
\arrayrulecolor{black}
\begin{tabular}{|p{0.1\linewidth}|p{0.18\linewidth}|p{0.22\linewidth}|p{0.20\linewidth}|p{0.22\linewidth}|}
\hline
\rowcolor{gray!60} 
\textbf{Versione} & \textbf{Data} & \textbf{Autore} & \textbf{Descrizione} & \textbf{Verificatore} \\
\hline
\rowcolor{white}
0.2.0 & 2025-11-05 & Giovanni Visentin & Aggiunti termini al documento & Gabriele Scaggiante \\
\hline
\rowcolor{gray!20}
0.1.0 & 2025-11-04 & Giovanni Visentin & Stesura iniziale del glossario & Gabriele Scaggiante \\
\hline
\end{tabular}
\end{center}

\end{document}
