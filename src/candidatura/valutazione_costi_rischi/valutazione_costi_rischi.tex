% template presente
\documentclass[a4paper,12pt]{article}

\usepackage[utf8]{inputenc}
\usepackage[italian]{babel}
\usepackage{graphicx}
\graphicspath{{resources/}{../resources/}{../../resources/}{../../../resources/}{../../../../resources/}}
\usepackage{xcolor}
\usepackage{geometry}
\usepackage{setspace}
\usepackage{colortbl}
\usepackage{hyperref} % link cliccabili
\usepackage{fancyhdr} % intestazioni personalizzate
\usepackage{titlesec} % per formattare le sezioni
\geometry{margin=2.5cm}

\setlength{\parindent}{0pt}
\setstretch{1.2}

% ===== Stile intestazione =====
\pagestyle{fancy}
\fancyhf{}
\fancyhead[L]{\textcolor{gray}{Preventivo Costi e Rischi Attesi - Gruppo 17}}
\fancyfoot[C]{\thepage}

% ===== Formattazione sezioni =====
\titleformat{\section}{\Large\bfseries}{\thesection.}{0.5em}{}
\titleformat{\subsection}{\large\bfseries}{\thesubsection.}{0.5em}{}

\begin{document}

% ======= HEADER UNIVERSITÀ E GRUPPO CENTRATI VERTICALMENTE =======
\vspace*{\fill} % --- Spinge verso il basso l'inizio del contenuto

\begin{center}
    \begin{minipage}{0.25\textwidth}
        \centering
        \includegraphics[width=\linewidth]{logoUni.png}
    \end{minipage}
    \hfill
    \begin{minipage}{0.7\textwidth}
        \raggedright
        {\color{red}\LARGE \textbf{Università degli Studi di Padova}}\\[0.3cm]
        {\large
        Laurea: Informatica\\
        Corso: Ingegneria del Software\\
        Anno Accademico: 2025/26
        }
    \end{minipage}
\end{center}

\vspace{1cm}

\begin{center}
    \begin{minipage}{0.25\textwidth}
        \centering
        \includegraphics[width=\linewidth]{logo.png}
    \end{minipage}
    \hfill
    \begin{minipage}{0.7\textwidth}
        \raggedright
        {\LARGE \textbf{Gruppo 17}}\\[0.3cm]
        {\large
        Nome: BitByBit\\
        Email: swe.bitbybit@gmail.com
        }
    \end{minipage}
\end{center}

\vspace{1.5cm}

\begin{center}
    {\LARGE \textbf{Preventivo Costi e Rischi Attesi}}
\end{center}

\vspace*{\fill} % --- Spinge verso l’alto la fine del contenuto, centrando tutto il blocco

\clearpage

% ======= INDICE SU PAGINA DEDICATA =======
\clearpage
\tableofcontents
\thispagestyle{empty} % senza numero di pagina per l'indice
\clearpage

% ---------- ANALISI RISCHI ----------
\section{Analisi dei Rischi}
\begin{itemize}
    \item \textbf{Rischio}: Numero considerevole di nuovi sistemi da imparare.
        \begin{itemize}
            \item \textbf{Mitigazione}: Programmare una fase iniziale di formazione interna e autoformazione con documentazione condivisa. Ogni membro si impegnerà ad approfondire un sottoinsieme specifico di strumenti e a condividerne le conoscenze con il resto del gruppo, riducendo così il tempo di apprendimento complessivo.
        \end{itemize}

    \item \textbf{Rischio}: Scarso impegno da parte di un membro del gruppo.
        \begin{itemize}
            \item \textbf{Mitigazione}: Stabilire fin da subito obiettivi individuali chiari e tracciabili tramite strumenti di project management. In caso di mancato rispetto delle scadenze, il responsabile interverrà con una redistribuzione temporanea dei compiti per garantire la continuità del progetto.
        \end{itemize}

    \item \textbf{Rischio}: Impegni personali e universitari.
        \begin{itemize}
            \item \textbf{Mitigazione}: Pianificare riunioni e obiettivi in anticipo, in modo da evitare di scoprire problemi di disponibilità in momenti critici. 
        \end{itemize}

    \item \textbf{Rischio}: Indecisioni all'interno del gruppo.
        \begin{itemize}
            \item \textbf{Mitigazione}: Nominare un responsabile di riferimento per le decisioni critiche e adottare un metodo decisionale basato sul consenso o, in caso di stallo, sul voto a maggioranza. Le scelte chiave saranno documentate nei verbali delle riunioni per evitare ambiguità future.
        \end{itemize}

    \item \textbf{Rischio}: Ritardi nella consegna delle funzionalità.
        \begin{itemize}
            \item \textbf{Mitigazione}: Suddividere il lavoro in sprint brevi con obiettivi incrementali verificabili. Monitorare i progressi tramite milestone e retrospettive per identificare tempestivamente eventuali blocchi o ritardi.
        \end{itemize}
    
    \item \textbf{Rischio}: Cambiamenti nei requisiti da parte del proponente.
        \begin{itemize}
            \item \textbf{Mitigazione}: Mantenere una comunicazione costante con il committente e documentare ogni variazione dei requisiti. Implementare il progetto in modo modulare per rendere più agevole l’adattamento a nuove richieste senza compromettere il lavoro già svolto.
        \end{itemize}
\end{itemize}

% ---------- TABELLA RIASSUNTIVA ----------
\vspace{0.5cm}
\subsection{Tabella Riassuntiva dei Rischi}

\begin{center}
    \small
    \renewcommand{\arraystretch}{1.2}
    \arrayrulecolor{black}
    \begin{tabular}{|p{0.35\linewidth}|p{0.15\linewidth}|p{0.10\linewidth}|p{0.30\linewidth}|}
        \hline
        \rowcolor{gray!60}
        \textbf{Rischio} & \textbf{Probabilità} & \textbf{Impatto} & \textbf{Mitigazione Principale} \\
        \hline
        \rowcolor{white}
        Nuovi sistemi da imparare & Media & Alto & Formazione interna e condivisione conoscenze \\
        \hline
        \rowcolor{gray!20}
        Scarso impegno di un membro & Bassa & Alto & Definizione di obiettivi individuali e monitoraggio \\
        \hline
        \rowcolor{white}
        Impegni personali e universitari & Alta & Medio & Pianificazione anticipata e calendario condiviso \\
        \hline
        \rowcolor{gray!20}
        Indecisioni nel gruppo & Media & Medio & Processo decisionale strutturato e verbali \\
        \hline
        \rowcolor{white}
        Ritardi nelle consegne & Media & Alto & Suddivisione in sprint e milestone \\
        \hline
        \rowcolor{gray!20}
        Cambiamenti dei requisiti & Bassa & Alto & Comunicazione costante e sviluppo modulare \\
        \hline
    \end{tabular}
\end{center}

% ---------- ANALISI RUOLI ----------
\section{Analisi dei Ruoli}
Per uno corretto svolgimento del progetto sono richiesti i seguenti ruoli:

\begin{itemize}
    \item \textbf{Responsabile}:
        \begin{itemize}
            \item Il responsabile è un ruolo importante per tutta la durate del progetto.
            Ha il compito di coordinare le attività del gruppo e pianificare e gestire le risorse del progetto, oltre a dover rappresentare il gruppo nei confronti di entità esterne.
        \end{itemize}

    \item \textbf{Amministratore}:
        \begin{itemize}
            \item L'amministratore si occupa della gestione dei sistemi informatici in uso attivo e uso futuro del progetto e della gestione di eventuali ticket dovuti a problemi nel funzionamento dell'infrastruttura.
            Il ruolo è rilevante per tutto lo svolgimento del progetto data la necessità di avere il maggior uptime possibile degli strumenti e sistemi utilizzati in esso.
        \end{itemize}

    \item \textbf{Analista}:
        \begin{itemize}
            \item Lo scopo dell'analista è quello di identificare i requisiti del progetto, interpretando le necessità degli utenti finali in modo da ottenere una corretta definizione delle funzionalità richieste.
            Le ore dedicate al ruolo di analista diminuiscono con l'avanzare del progetto, ma il ruolo rimane attivo per eventuali aggiornamenti dei requisiti in seguito a eventuali confronti con il proponente.
        \end{itemize}

    \item \textbf{Progettista}:
        \begin{itemize}
            \item Il ruolo di progettista si occupa di tradurre i requisiti identificati dagli analisti in qualcosa di implementabile, e ne supervisiona l'implementazione da parte dei programmatori.
        \end{itemize}

    \item \textbf{Programmatore}:
        \begin{itemize}
            \item I programmatori si occupano della effettiva realizzazione della parte software del progetto, collaborando con i progettisti per assicurare una corretta implementazione di tutte le funzionalità richieste.
        \end{itemize}

    \item \textbf{Verificatore}:
        \begin{itemize}
            \item Il ruolo di verificatore si assicura che la qualità dei prodotti e processi adottati venga mantenuta per tutta la durata del progetto, metiante revisioni e test.
        \end{itemize}
\end{itemize}

% ---------- IMPEGNO ORARIO PREVISTO E ROTAZIONE RUOLI ----------
\section{Impegno Orario Previsto e Rotazione Ruoli}
In una riunione tra i membri del gruppo sono state decise le ore dedicate al progetto da parte di ciascun membro, che verranno suddivise nei ruoli identificati nella sezione precedente.
La tabella seguente indica la disponibilità oraria di ciascun membro.
\begin{center}
    \small
    \renewcommand{\arraystretch}{1.2} 
    \arrayrulecolor{black} 
    \begin{tabular}{|p{0.30\linewidth}|p{0.20\linewidth}|}
        \hline
        \rowcolor{gray!60} 
        \textbf{Persona} & \textbf{Ore Assegnate}\\
        \hline
        \rowcolor{white}
        Giovanni Visentin & 90 \\
        \hline
        \rowcolor{gray!20}
        Dennis Parolin & 90 \\
        \hline
        \rowcolor{white}
        Riccardo Manisi	& 90 \\
        \hline
        \rowcolor{gray!20}
        Ferdinando Fracasso & 90 \\
        \hline
        \rowcolor{white}
        Gabriele Scaggiante & 90 \\
        \hline
        \rowcolor{gray!20}
        Marco Sanguin & 90 \\
        \hline
        \rowcolor{gray!60}
        \textbf{Totale} & 540 \\
        \hline
    \end{tabular}
\end{center}
    
Inoltre sono stimate le seguenti ore medie per ruolo per ciascun membro.
\space
\begin{center}
    \small
    \renewcommand{\arraystretch}{1.2}
    \arrayrulecolor{black} 
    \begin{tabular}{|p{0.20\linewidth}|p{0.30\linewidth}|}
        \hline
        \rowcolor{gray!60}
        \textbf{Ruolo} & \textbf{Ore medie per persona}\\
        \hline
        \rowcolor{white}
        Responsabile & 13 \\
        \hline
        \rowcolor{gray!20}
        Amministratore & 14 \\
        \hline
        \rowcolor{white}
        Analista & 11 \\
        \hline
        \rowcolor{gray!20}
        Progettista & 17 \\
        \hline
        \rowcolor{white}
        Programmatore & 20 \\
        \hline
        \rowcolor{gray!20}
        Verificatore & 15 \\
        \hline
    \end{tabular}
\end{center}

\subsection{Rotazione dei ruoli}
Per la durata del progetto è previsto che i ruoli dei membri ruotino in corrispondenza agli sprint, fino all'esaurimento delle ore produttive assegnate. 

% ---------- PREVENTIVO COSTI ----------
\section{Preventivo Costi}
In seguito è riportata una tabella contente le ore assegnate a ciascun ruolo, il relativo costo orario per ruolo ed il conseguente costo totale, sempre diviso per ruolo.
\begin{center}
    \small
    \renewcommand{\arraystretch}{1.2} 
    \arrayrulecolor{black} 
    \begin{tabular}{|p{0.20\linewidth}|p{0.10\linewidth}|p{0.25\linewidth}|p{0.20\linewidth}|}
        \hline
        \rowcolor{gray!60} 
        \textbf{Ruolo} & \textbf{Ore} & \textbf{Costo Orario} & \textbf{Costo Totale} \\
        \hline
        \rowcolor{white}
        Responsabile & 78 & 30,00€/h & 2.340,00€\\
        \hline
        \rowcolor{gray!20}
        Amministratore & 84 & 20,00€/h & 1.680,00€\\
        \hline
        \rowcolor{white}
        Analista & 66 & 25,00€/h & 1.650,00€\\
        \hline
        \rowcolor{gray!20}
        Progettista & 102 & 25,00€/h & 2.550,00€\\
        \hline
        \rowcolor{white}
        Programmatore & 120 & 15,00€/h & 1.800,00€\\
        \hline
        \rowcolor{gray!20}
        Verificatore & 90 & 15,00€/h & 1.350,00€\\
        \hline
    \end{tabular}
\end{center}
È previsto un impegno collettivo totale di 540 ore, corrispondente ad un costo totale preventivato di 11.370,00€. 

% ---------- DATA CONSEGNA ----------
\section{Data di Consegna}
La data di consegna prevista per il progetto è il 2026-04-15.

% ---------- REDAZIONE E REVISIONE ----------
\clearpage
\section{Redazione e revisioni del documento}

\begin{center}
\small
\renewcommand{\arraystretch}{1.2} 
\arrayrulecolor{black}
\begin{tabular}{|p{0.1\linewidth}|p{0.12\linewidth}|p{0.22\linewidth}|p{0.26\linewidth}|p{0.22\linewidth}|}
\hline
\rowcolor{gray!60} 
\textbf{Versione} & \textbf{Data} & \textbf{Autore} & \textbf{Descrizione} & \textbf{Verificatore} \\
\hline
\rowcolor{white}
1.0.0 & 2025-11-04 & Fracasso Ferdinando & Aggiunte informazioni riguardanti la rotazione dei ruoli e modificata la data di consegna del progetto & Dennis Parolin \\
\hline
\rowcolor{gray!20}
0.3.0 & 2025-10-31 & Fracasso Ferdinando & Aggiunta tabella ripartizione ore & Dennis Parolin \\
\hline
\rowcolor{white}
0.2.0 & 2025-10-31 & Manisi Riccardo & Aggiunte mitigazioni e tabella riassuntiva & Dennis Parolin \\
\hline
\rowcolor{gray!20}
0.1.0 & 2025-10-30 & Fracasso Ferdinando & Stesura iniziale del documento & Dennis Parolin \\
\hline
\end{tabular}
\end{center}

\end{document}
