% template presente
\documentclass[a4paper,12pt]{article}

\usepackage[utf8]{inputenc}
\usepackage[italian]{babel}
\usepackage{graphicx}
\graphicspath{{resources/}{../resources/}{../../resources/}{../../../resources/}{../../../../resources/}}
\usepackage{xcolor}
\usepackage{geometry}
\usepackage{setspace}
\usepackage{colortbl}
\usepackage{hyperref}
\hypersetup{
    colorlinks=true,
    linkcolor=blue,
    filecolor=magenta,
    urlcolor=cyan,     
}
\usepackage{fancyhdr} 
\usepackage{titlesec}
\geometry{margin=2.5cm}

\setlength{\parindent}{0pt}
\setstretch{1.2}

% ===== Stile intestazione =====
\pagestyle{fancy}
\fancyhf{}
\fancyhead[L]{\textcolor{gray}{Documento Candidatura - Gruppo 17}}
\fancyfoot[C]{\thepage}

% ===== Formattazione sezioni =====
\titleformat{\section}{\Large\bfseries}{\thesection.}{0.5em}{}
\titleformat{\subsection}{\large\bfseries}{\thesubsection.}{0.5em}{}

\begin{document}

% ======= HEADER UNIVERSITÀ E GRUPPO CENTRATI VERTICALMENTE =======
\vspace*{\fill} % --- Spinge verso il basso l'inizio del contenuto

\begin{center}
    \begin{minipage}{0.25\textwidth}
        \centering
        \includegraphics[width=\linewidth]{logoUni.png}
    \end{minipage}
    \hfill
    \begin{minipage}{0.7\textwidth}
        \raggedright
        {\color{red}\LARGE \textbf{Università degli Studi di Padova}}\\[0.3cm]
        {\large
        Laurea: Informatica\\
        Corso: Ingegneria del Software\\
        Anno Accademico: 2025/26
        }
    \end{minipage}
\end{center}

\vspace{1cm}

\begin{center}
    \begin{minipage}{0.25\textwidth}
        \centering
        \includegraphics[width=\linewidth]{logo.png}
    \end{minipage}
    \hfill
    \begin{minipage}{0.7\textwidth}
        \raggedright
        {\LARGE \textbf{Gruppo 17}}\\[0.3cm]
        {\large
        Nome: BitByBit\\
        Email: swe.bitbybit@gmail.com
        }
    \end{minipage}
\end{center}

\vspace*{1cm}

\begin{flushleft}
\textbf{Destinatari:}\\
Prof. Tullio Vardanega\\
Prof. Riccardo Cardin\\
\end{flushleft}

\vspace{1cm}

\vspace{1.5cm}

\begin{center}
    {\LARGE \textbf{Documento Candidatura}}
\end{center}

\vspace*{\fill}

\clearpage

% ======= INDICE SU PAGINA DEDICATA =======
\clearpage
\tableofcontents
\thispagestyle{empty} % senza numero di pagina per l'indice
\clearpage


\section{Candidatura}
Il gruppo \textbf{BitByBit} intende ufficialmente candidarsi per lo sviluppo del capitolato \textbf{C8 - "Smart order"}, proposto dall’azienda \textbf{Ergon Informatica s.r.l.} .

La decisione è stata presa a seguito di un’analisi approfondita dei capitolati proposti, valutando in particolare:
\begin{itemize}
    \item l’interesse tecnico e formativo del progetto;
    \item la fattibilità rispetto alle competenze del gruppo.
\end{itemize}


Le motivazioni che hanno portato il gruppo \textbf{BitByBit} a selezionare il capitolato \textbf{C8 – Smart Order} proposto da \textbf{Ergon Informatica s.r.l.} sono illustrate nel documento dedicato, consultabile al seguente link:  
\href{https://github.com/SWE-BitByBit/SWE-project/blob/d1ed2df7ab879a4755811d7b17a54d737cb9af04/candidatura/Valutazione_capitolati.pdf}{\textbf{Analisi dei Capitolati}}.

Per ulteriori approfondimenti tecnici e per la visione completa della documentazione aggiornata, è possibile consultare la \href{https://github.com/SWE-BitByBit/SWE-project.git}{\textbf{repository ufficiale del progetto}} su GitHub.

% ===== INFO INCONTRI CON PROPONENTI =====
\subsection{Incontri con le aziende proponenti}

Nel corso della fase di analisi preliminare, il gruppo ha organizzato diversi incontri con le aziende proponenti per ottenere chiarimenti e specifiche aggiuntive sui capitolati di interesse.  
Ogni incontro è stato verbalizzato e i verbali sono stati firmati dalle rispettive aziende per attestare l’avvenuto confronto.

\begin{itemize}
    \item \textbf{Zucchetti S.p.A.} – approfondimento sul capitolato \emph{C6 Second Brain}.  
    Verbale disponibile al seguente link: \href{<LINK_VERBALE_ZUCCHETTI>}{Verbale esterno 2025-10-23}.
    \item \textbf{Ergon Informatica s.r.l.} – discussione sul capitolato \emph{C8 Smart Order}.  
    Verbale disponibile al seguente link: \href{<LINK_VERBALE_ERGON>}{Verbale esterno 2025-10-27}.
    \item \textbf{Miriade S.p.A.} – incontro per chiarimenti sul capitolato \emph{C4 L’app che protegge e trasforma}.  
    Verbale disponibile al seguente link: \href{<LINK_VERBALE_MIRIADE>}{Verbale esterno 2025-10-27}.
\end{itemize}

% ===== PREVENTIVO COSTI ======
\subsection{Preventivo costi e rischi}
Come indicato nel documento \href{URL}{Documento costi e rischi} nel repository GitHub il costo stimato dal gruppo per la realizzazione del capitolato ammonta a \textbf{11.100,00€}, con la consegna prevista in data \textbf{2026-03-08}.

% ===== COMPOSIZIONE DEL GRUPPO =====
\subsection{Composizione del gruppo}

\begin{center}
\small
\renewcommand{\arraystretch}{1.2}
\begin{tabular}{|p{0.3\linewidth}|p{0.2\linewidth}|}
\hline
\rowcolor{gray!60}
\textbf{Nome e Cognome} & \textbf{Matricola} \\
\hline
Dennis Parolin & 2113203 \\
\hline
Ferdinando Fracasso & 2122649 \\
\hline
Giovanni Visentin & 2101064 \\
\hline
Riccardo Manisi & 2111948 \\
\hline
Marco Sanguin & 2103121 \\
\hline
Gabriele Scaggiante & 2101076 \\
\hline
\end{tabular}
\end{center}

% ---------- REDAZIONE E REVISIONE ----------
\clearpage
\section{Redazione e revisioni del documento}

\begin{center}
\small
\renewcommand{\arraystretch}{1.2} 
\arrayrulecolor{black}
\begin{tabular}{|p{0.1\linewidth}|p{0.12\linewidth}|p{0.22\linewidth}|p{0.26\linewidth}|p{0.22\linewidth}|}
\hline
\rowcolor{gray!60} 
\textbf{Versione} & \textbf{Data} & \textbf{Autore} & \textbf{Descrizione} & \textbf{Verificatore} \\
\hline
\rowcolor{white}
1.0.0 & 2025-10-31 & Manisi Riccardo  & Aggiunta sezione costi e rischi, motivazioni scelta e link principale repository & Visentin Giovanni \\
\hline
\rowcolor{gray!20}
0.2.1 & 2025-10-31 & Manisi Riccardo & Aggiunta destinatari & Visentin Giovanni\\
\hline
\rowcolor{white}
0.2.0 & 2025-10-30 & Visentin Giovanni & Scrittura capitolato & Manisi Riccardo \\
\hline
\rowcolor{gray!20}
0.1.0 & 2025-10-30 & Visentin Giovanni & Stesura iniziale del verbale & Manisi Riccardo \\
\hline

\end{tabular}
\end{center}

\end{document}
