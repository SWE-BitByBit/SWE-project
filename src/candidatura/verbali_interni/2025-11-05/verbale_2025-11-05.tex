% template presente
\documentclass[a4paper,12pt]{article}

\usepackage{tabularx}
\usepackage{tcolorbox}
\usepackage[utf8]{inputenc}
\usepackage[italian]{babel}
\usepackage{hyperref}
\hypersetup{
	colorlinks=true,
	linkcolor=blue,
	filecolor=blue,
	urlcolor=blue,     
}
\usepackage{graphicx}
\graphicspath{{resources/}{../resources/}{../../resources/}{../../../resources/}{../../../../resources/}}
\usepackage{xcolor}
\usepackage{geometry}
\usepackage{setspace}
\usepackage{colortbl}
\usepackage{hyperref} 
\usepackage{fancyhdr} 
\usepackage{titlesec}
\geometry{margin=2.5cm}
% ===== Variabile per il titolo del documento =====
\newcommand{\documenttitle}{Verbale Riunione Interna}


\setlength{\parindent}{0pt}
\setstretch{1.2}

% ===== Stile intestazione =====
\pagestyle{fancy}
\fancyhf{}
\fancyhead[L]{\textcolor{gray}{\documenttitle - BitByBit}}
\fancyfoot[C]{\thepage}

% ===== Formattazione sezioni =====
\titleformat{\section}{\Large\bfseries}{\thesection.}{0.5em}{}
\titleformat{\subsection}{\large\bfseries}{\thesubsection.}{0.5em}{}

\begin{document}
	
	% ======= HEADER UNIVERSITÀ E GRUPPO CENTRATI VERTICALMENTE =======
	\vspace*{\fill} % --- Spinge verso il basso l'inizio del contenuto
	
	\begin{center}
		\begin{minipage}{0.25\textwidth}
			\centering
			\includegraphics[width=\linewidth]{logoUni.png}
		\end{minipage}
		\hfill
		\begin{minipage}{0.7\textwidth}
			\raggedright
			{\color{red}\LARGE \textbf{Università degli Studi di Padova}}\\[0.3cm]
			{\large
				Laurea: Informatica\\
				Corso: Ingegneria del Software\\
				Anno Accademico: 2025/26
			}
		\end{minipage}
	\end{center}
	
	\vspace{1cm}
	
	\begin{center}
		\begin{minipage}{0.25\textwidth}
			\centering
			\includegraphics[width=\linewidth]{logo.png}
		\end{minipage}
		\hfill
		\begin{minipage}{0.7\textwidth}
			\raggedright
			{\LARGE \textbf{Gruppo 17}}\\[0.3cm]
			{\large
				Nome: BitByBit\\
				Email: swe.bitbybit@gmail.com
			}
		\end{minipage}
	\end{center}
	
	\vspace{1.5cm}
	
	\begin{center}
		{\LARGE \textbf{\documenttitle}}
	\end{center}
	
	\vspace*{\fill} % --- Spinge verso l’alto la fine del contenuto, centrando tutto il blocco
	
	\clearpage
	% ======= INFO GENERALI =======
	\section*{Informazioni Generali}
	\renewcommand{\arraystretch}{1.3}
	
	\begin{tcolorbox}
		\begin{tabularx}{\textwidth}{@{}l X@{}}
			\textbf{Redattore:} & Sanguin Marco \\
			\textbf{Verificatore:} & Visentin Giovanni \\
			\textbf{Data:} &  2025-11-05 \\
			\textbf{Durata:} & 1h 45m \\
			\textbf{Luogo:} & Discord \\
		\end{tabularx}
	\end{tcolorbox}
	
	\vspace{0.4cm}
	\textbf{Partecipanti:}
	\begin{itemize}
		\item Manisi Riccardo
		\item Scaggiante Gabriele
		\item Sanguin Marco
		\item Visentin Giovanni
		\item Fracasso Ferdinando
	\end{itemize}
	
	\textbf{Assenti:}
	\begin{itemize}
		\item Parolin Dennis
	\end{itemize}
	
	\clearpage
	
	\vspace{0.5cm}
	
	\vspace{0.8cm}
	
	% ======= INDICE SU PAGINA DEDICATA =======
	\clearpage
	\tableofcontents
	\thispagestyle{empty} % senza numero di pagina per l'indice
	\clearpage
	
	% ---------- ORDINE DEL GIORNO ----------
	\section{Ordine del Giorno}
	\begin{itemize}
		\item Confermare il cambio di capitolato da C8 a C4, a seguito dell'assegnazione ad un altro gruppo.
		\item Completare e aggiornare i documenti per l'invio della seconda candidatura e sistemare il sito.
	\end{itemize}
	
	% ---------- DISCUSSIONI ----------
	\section{Discussioni}
	Durante la riunione interna, i membri del gruppo hanno concordato di mandare la richiesta di candidatura al capitolato C4, cambiando da C8 in quanto gi\`a assegnato ad un altro gruppo; quindi hanno discusso sui documenti che andavano redatti e come si sarebbero potuti migliorare per renderli pi\`u ordinati e comprensibili, aggiornando anche la disposizione dei file e delle cartelle nella repository del progetto e sistemando i piccoli errori nel sito.
	
	% ---------- DECISIONI ----------
	\section{Decisioni}
	Si \`e deciso di suddividersi i compiti in modo da sistemare tutti i documenti il prima possibile, quindi di redarre la nuova domanda di candidatura e di inviarla entro la giornata di domani, posticipando la consegna a causa del cambio di capitolato.
	
	% ---------- TO DO -----------
	\section{To Do}
	Dalle discussioni e decisioni intraprese, sono sorte le seguenti task:
	
	\begin{center}
		\small
		\renewcommand{\arraystretch}{1.2} 
		\arrayrulecolor{black} 
		\begin{tabular}{|p{0.35\linewidth}|p{0.2\linewidth}|p{0.2\linewidth}|p{0.2\linewidth}|}
			\hline
			\rowcolor{gray!60} 
			\textbf{Task} & \textbf{Assegnatario} & \textbf{Scadenza} & \textbf{Link GitHub} \\
			\hline
			\rowcolor{white}
			Redarre il documento di valutazione capitolati per la seconda candidatura & Gabriele Scaggiante & 2025-11-05 & \href{https://github.com/SWE-BitByBit/SWE-project/issues/54}{\textbf{\#54}} \\
			\hline
			\rowcolor{gray!20}
			Redarre la domanda di seconda candidatura & Visentin Giovanni & 2025-11-05 & \href{https://github.com/SWE-BitByBit/SWE-project/issues/50}{\textbf{\#50}} \\
			\hline
			\rowcolor{white}
			Sistemare i link ai documenti nella pagina web & Visentin Giovanni & 2025-11-05 & \href{https://github.com/SWE-BitByBit/SWE-project/issues/49}{\textbf{\#49}} \\
			\hline
			\rowcolor{gray!20}
			Normalizzare i nome di file e cartelle & Riccardo Manisi & 2025-11-05 & \href{https://github.com/SWE-BitByBit/SWE-project/issues/53}{\textbf{\#53}} \\
			\hline
		\end{tabular}
	\end{center}
	
	% ---------- REDAZIONE E REVISIONE ----------
	\clearpage
	\section{Redazione e revisioni del documento}
	
	\begin{center}
		\small
		\renewcommand{\arraystretch}{1.2} 
		\arrayrulecolor{black}
		\begin{tabular}{|p{0.1\linewidth}|p{0.18\linewidth}|p{0.22\linewidth}|p{0.20\linewidth}|p{0.22\linewidth}|}
			\hline
			\rowcolor{gray!60} 
			\textbf{Versione} & \textbf{Data} & \textbf{Autore} & \textbf{Descrizione} & \textbf{Verificatore} \\
			\hline
			\rowcolor{white}
			1.0.0 & 2025-11-05 & Marco Sanguin  & Stesura iniziale del verbale & Visentin Giovanni \\
			\hline
			\rowcolor{gray!20} 
			1.0.1 & 2025-11-05 & Gabriele Scaggiante & Rinominazione issues & Visentin Giovanni \\
			\hline
	
		\end{tabular}
	\end{center}
	
\end{document}
