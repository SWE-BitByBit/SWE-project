% template presente
\documentclass[a4paper,12pt]{article}

\usepackage[utf8]{inputenc}
\usepackage[italian]{babel}
\usepackage{graphicx}
\graphicspath{{resources/}{../resources/}{../../resources/}{../../../resources/}{../../../../resources/}}
\usepackage{xcolor}
\usepackage{geometry}
\usepackage{setspace}
\usepackage{colortbl}
\usepackage{hyperref} % link cliccabili
\usepackage{fancyhdr} % intestazioni personalizzate
\usepackage{titlesec} % per formattare le sezioni
\geometry{margin=2.5cm}

\setlength{\parindent}{0pt}
\setstretch{1.2}

% ===== Stile intestazione =====
\pagestyle{fancy}
\fancyhf{}
\fancyhead[L]{\textcolor{gray}{Verbale Riunione - Gruppo 17}}
\fancyfoot[C]{\thepage}

% ===== Formattazione sezioni =====
\titleformat{\section}{\Large\bfseries}{\thesection.}{0.5em}{}
\titleformat{\subsection}{\large\bfseries}{\thesubsection.}{0.5em}{}

\begin{document}

% ======= HEADER UNIVERSITÀ E GRUPPO CENTRATI VERTICALMENTE =======
\vspace*{\fill} % --- Spinge verso il basso l'inizio del contenuto

\begin{center}
    \begin{minipage}{0.25\textwidth}
        \centering
        \includegraphics[width=\linewidth]{logoUni.png}
    \end{minipage}
    \hfill
    \begin{minipage}{0.7\textwidth}
        \raggedright
        {\color{red}\LARGE \textbf{Università degli Studi di Padova}}\\[0.3cm]
        {\large
        Laurea: Informatica\\
        Corso: Ingegneria del Software\\
        Anno Accademico: 2025/26
        }
    \end{minipage}
\end{center}

\vspace{1cm}

\begin{center}
    \begin{minipage}{0.25\textwidth}
        \centering
        \includegraphics[width=\linewidth]{logo.png}
    \end{minipage}
    \hfill
    \begin{minipage}{0.7\textwidth}
        \raggedright
        {\LARGE \textbf{Gruppo 17}}\\[0.3cm]
        {\large
        Nome: BitByBit\\
        Email: swe.bitbybit@gmail.com
        }
    \end{minipage}
\end{center}

\vspace{1.5cm}

\begin{center}
    {\LARGE \textbf{Verbale Riunione Numero X}}
\end{center}

\vspace*{\fill} % --- Spinge verso l’alto la fine del contenuto, centrando tutto il blocco

\clearpage
% ======= INFO GENERALI =======
{\large \textbf{Informazioni generali}}
{\footnotesize
\setstretch{1.1}

\begin{itemize}
    \item \textbf{Data:} 2025-10-20
    \item \textbf{Ora inizio:} 15.30
    \item \textbf{Ora fine:} 17.30
    \item \textbf{Tipo riunione:} Interna
    \item \textbf{Luogo:} Discord
    \item \textbf{Durata:} 2h
    \item \textbf{Responsabile:} Dennis Parolin
\end{itemize}

\vspace{0.2cm}

\textbf{Partecipanti:}
\begin{itemize}
    \item Dennis Parolin
    \item Riccardo Manisi
    \item Gabriele Scaggiante
    \item Ferdinando Fracasso
    \item Giovanni Visentin
\end{itemize}

\textbf{Assenti:}
\begin{itemize}
    \item Nessuno (manca il quinto membro ancora non definito)
\end{itemize}
}

\vspace{0.5cm}

\vspace{0.8cm}

% ======= INDICE SU PAGINA DEDICATA =======
\clearpage
\tableofcontents
\thispagestyle{empty} % senza numero di pagina per l'indice
\clearpage

% ---------- ORDINE DEL GIORNO ----------
\section{Ordine del Giorno}
\begin{itemize}
    \item Incontro formativo effettivo del gruppo (solo ora considerato il problema noto)
    \item Discussione sugli appalti
    \item Riorganizzazione della repository github
\end{itemize}

% ---------- DISCUSSIONI ----------
\section{Discussioni}
Il gruppo ha deciso in modo definitivo le 3 preferenze di progetti. Sono sorti dubbi riguardante questi 3 progetti, alcuni sono stati chiariti parlando tra componenti del gruppo, altri invece sono rimasti. Per questo motivo è stata inviata una mail al referente del progetto (quello della scelta primaria) chiedendo un eventuale incontro. Infine è stata spiegata l'attuale struttura della repository github (che era già stata creata al di fuori dai meeting) e riorganizzata/migliorata in parte secondo quanto sorto da una breve discussione riguardante tale argomento.

% ---------- DECISIONI ----------
\section{Decisioni}
È stato deciso di di tenere come prima preferenza l'appalto C6, come seconda scelta abbiamo deciso l'appalto C8 e come terza scelta l'appalto C9

% ---------- TO DO ----------
\section{To Do}
Dalle discussioni e decisioni intraprese, sono sorte le seguenti task:

\begin{center}
\small
\renewcommand{\arraystretch}{1.2} 
\arrayrulecolor{black} 
\begin{tabular}{|p{0.45\linewidth}|p{0.25\linewidth}|p{0.25\linewidth}|}
\hline
\rowcolor{gray!60} 
\textbf{Task} & \textbf{Assegnatario} & \textbf{Scadenza} \\
\hline
\rowcolor{white}
Migliorare ulteriormente la repository di github (aggiungere action per i file LaTeX) & Riccardo Manisi & 2025-10-26 \\
\hline
\rowcolor{white}
Chiedere incontri per chiarire alcuni dubbi anche sui due appalti non di preferenza primaria & Tutto il gruppo & 2025-10-27 \\
\hline
\end{tabular}
\end{center}

% ---------- REDAZIONE E REVISIONE ----------
\clearpage
\section{Redazione e revisioni del documento}

\begin{center}
\small
\renewcommand{\arraystretch}{1.2} 
\arrayrulecolor{black}
\begin{tabular}{|p{0.1\linewidth}|p{0.12\linewidth}|p{0.22\linewidth}|p{0.26\linewidth}|p{0.22\linewidth}|}
\hline
\rowcolor{gray!60} 
\textbf{Versione} & \textbf{Data} & \textbf{Autore} & \textbf{Descrizione} & \textbf{Verificatore} \\
\hline
\rowcolor{white}
1.0.0 & 2025-10-22 & Dennis Parolin & Aggiunta tabella di redazione e revisione & Ferdinando Fracasso \\
\hline
\rowcolor{gray!20}
0.1.0 & 2025-10-20 & Dennis Parolin & Stesura iniziale del verbale & Riccardo Manisi \\
\hline
\end{tabular}
\end{center}

\end{document}
