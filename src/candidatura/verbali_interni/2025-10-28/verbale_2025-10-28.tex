% template presente
\documentclass[a4paper,12pt]{article}

\usepackage[utf8]{inputenc}
\usepackage[italian]{babel}
\usepackage{graphicx}
\graphicspath{{resources/}{../resources/}{../../resources/}{../../../resources/}{../../../../resources/}}
\usepackage{xcolor}
\usepackage{geometry}
\usepackage{setspace}
\usepackage{colortbl}
\usepackage{hyperref} % link cliccabili
\usepackage{fancyhdr} % intestazioni personalizzate
\usepackage{titlesec} % per formattare le sezioni
\geometry{margin=2.5cm}

\setlength{\parindent}{0pt}
\setstretch{1.2}

% ===== Stile intestazione =====
\pagestyle{fancy}
\fancyhf{}
\fancyhead[L]{\textcolor{gray}{Verbale Riunione - Gruppo 17}}
\fancyfoot[C]{\thepage}

% ===== Formattazione sezioni =====
\titleformat{\section}{\Large\bfseries}{\thesection.}{0.5em}{}
\titleformat{\subsection}{\large\bfseries}{\thesubsection.}{0.5em}{}

\begin{document}

% ======= HEADER UNIVERSITÀ E GRUPPO CENTRATI VERTICALMENTE =======
\vspace*{\fill} % --- Spinge verso il basso l'inizio del contenuto

\begin{center}
    \begin{minipage}{0.25\textwidth}
        \centering
        \includegraphics[width=\linewidth]{logoUni.png}
    \end{minipage}
    \hfill
    \begin{minipage}{0.7\textwidth}
        \raggedright
        {\color{red}\LARGE \textbf{Università degli Studi di Padova}}\\[0.3cm]
        {\large
        Laurea: Informatica\\
        Corso: Ingegneria del Software\\
        Anno Accademico: 2025/26
        }
    \end{minipage}
\end{center}

\vspace{1cm}

\begin{center}
    \begin{minipage}{0.25\textwidth}
        \centering
        \includegraphics[width=\linewidth]{logo.png}
    \end{minipage}
    \hfill
    \begin{minipage}{0.7\textwidth}
        \raggedright
        {\LARGE \textbf{Gruppo 17}}\\[0.3cm]
        {\large
        Nome: BitByBit\\
        Email: swe.bitbybit@gmail.com
        }
    \end{minipage}
\end{center}

\vspace{1.5cm}

\begin{center}
    {\LARGE \textbf{Verbale Riunione Interna}}
\end{center}

\vspace*{\fill} % --- Spinge verso l’alto la fine del contenuto, centrando tutto il blocco

\clearpage
% ======= INFO GENERALI =======
{\large \textbf{Informazioni generali}}
{\footnotesize
\setstretch{1.1}

\begin{itemize}
    \item \textbf{Data:} 2025-10-28
    \item \textbf{Ora inizio:} 14:30
    \item \textbf{Ora fine:} 16:45
    \item \textbf{Tipo riunione:} Interna
    \item \textbf{Luogo:} Discord
    \item \textbf{Durata:} 2 h 15 min
    \item \textbf{Responsabile:} Gabriele Scaggiante
\end{itemize}

\vspace{0.2cm}

\textbf{Partecipanti:}
\begin{itemize}
    \item Gabriele Scaggiante
    \item Giovanni Visentin
    \item Dennis Parolin 
    \item Riccardo Manisi
    \item Ferdinando Fracasso
\end{itemize}

\textbf{Assenti:}
\begin{itemize}
    \item Marco Sanguin
\end{itemize}
}

\vspace{0.5cm}

\vspace{0.8cm}

% ======= INDICE SU PAGINA DEDICATA =======
\clearpage
\tableofcontents
\thispagestyle{empty} % senza numero di pagina per l'indice
\clearpage

% ---------- ORDINE DEL GIORNO ----------
\section{Ordine del Giorno}
\begin{itemize}
    \item Riepilogo delle riunioni esterne fatte
    \item Raccolta di opinioni generali per proporre la candidatura
\end{itemize}

% ---------- DISCUSSIONI ----------
\section{Discussioni}
Durante la riunione interna, i membri del gruppo hanno discusso in merito alla scelta del progetto da realizzare, analizzando preferenze personali, complessità tecniche e disponibilità dei referenti.

\subsection{Preferenze dei membri}
Ciascun componente ha espresso la propria preferenza tra i progetti proposti:
\begin{itemize}
    \item \textbf{Riccardo} e \textbf{Giovanni} hanno manifestato interesse per il progetto \textbf{C4}, principalmente per l’aspetto legato al design, all’accessibilità e alla possibilità di sviluppare un’applicazione mobile.
    \item \textbf{Gabriele}, \textbf{Dennis} e \textbf{Ferdinando} preferiscono il progetto \textbf{C8}, considerandolo più gestibile e vicino alle competenze già acquisite, pur trattando argomenti rilevanti e utili alla formazione.
\end{itemize}
Chi ha espresso preferenza per il C8 ha evidenziato che il C4 presenta una maggiore complessità tecnica e richiederebbe un impegno più elevato in termini di tempo e risorse.  
\\
Il capitolato n.6 per il momento è stato escluso: tutti i partecipanti alla riunione esterna hanno infatti espresso opinioni negative in merito alla direzione presa dal progetto, che sembrava più interessante prima dell'incontro. Anche il referente non ha dato un'impressione positiva.

\subsection{Considerazioni sui progetti C4 e C8}
È stato osservato che il progetto C8 risulta già assegnato, perciò la scelta potrebbe comportare difficoltà nel processo di selezione.  
Se il progetto C4 fosse meno impegnativo, tutti i membri sarebbero favorevoli alla sua realizzazione.  

I referenti del progetto C4 sono stati giudicati positivamente da parte di Dennis, che li ha trovati disponibili e collaborativi, mentre Gabriele ha percepito un atteggiamento più distaccato e non particolarmente professionale.

\subsection{Disponibilità e modalità di lavoro}
Sono state discusse le modalità di collaborazione con le aziende:
\begin{itemize}
    \item I referenti del C4 sono stati percepiti da Dennis come poco disponibili a incontri frequenti.
    \item Gabriele ha suggerito di favorire l’apprendimento individuale per velocizzare i tempi, evitando di coinvolgere tutto il gruppo in ogni fase operativa.
\end{itemize}
In termini di utilizzo dell’intelligenza artificiale, entrambi i progetti — C4 e C8 — sembrano offrire opportunità equivalenti, con la differenza che il C4 consentirebbe di approfondire maggiormente lo sviluppo mobile.

\subsection{Analisi tecnica del progetto C4}
Per la realizzazione del progetto C4 sarebbero necessarie competenze e attività nei seguenti ambiti:
\begin{itemize}
    \item Linguaggi e framework: \textbf{Flutter} e \textbf{Dart};
    \item Design dell’applicazione (UX, funzionalità, onboarding) — stimato in circa due settimane di lavoro;
    \item Infrastruttura e servizi cloud: \textbf{AWS} (Amazon Bedrock, database SQL e NoSQL);
    \item Contenimento e distribuzione: \textbf{Docker};
    \item Aspetti legali: conformità \textbf{GDPR} e diritto all’oblio;
    \item Interazioni con il dispositivo (notifiche, microfono, accesso tramite password, ecc.).
\end{itemize}

\subsection{Gestione della complessità e degli obiettivi}
È stato proposto di consultare il \textbf{Professor Vardanega} per valutare la gestione del carico di lavoro nel caso in cui si scelga il C4.  
Si è inoltre sottolineato che gli obiettivi del progetto potranno essere eventualmente ridimensionati in corso d’opera, pur consapevoli che una riduzione eccessiva delle funzionalità potrebbe compromettere il valore formativo del lavoro.  
Gabriele ha espresso preoccupazione riguardo al rischio di iniziare il progetto senza un quadro chiaro delle funzionalità da implementare sin dall’inizio.


% ---------- DECISIONI ----------
\section{Decisioni}
Al termine della discussione, non è stata ancora presa una decisione definitiva sul progetto da scegliere.  
Il gruppo intende contattare il \textbf{Professor Vardanega} per un confronto e predisporre una lista di domande da presentare in data \textbf{giovedì 30 ottobre 2025}.

% ---------- TO DO ----------
\section{To Do}
Annotare una breve lista di domande da porre al Professor Vardanega.

% ---------- REDAZIONE E REVISIONE ----------
\clearpage
\section{Redazione e revisioni del documento}

\begin{center}
\small
\renewcommand{\arraystretch}{1.2} 
\arrayrulecolor{black}
\begin{tabular}{|p{0.1\linewidth}|p{0.12\linewidth}|p{0.22\linewidth}|p{0.26\linewidth}|p{0.22\linewidth}|}
\hline
\rowcolor{gray!60} 
\textbf{Versione} & \textbf{Data} & \textbf{Autore} & \textbf{Descrizione} & \textbf{Verificatore} \\
\hline
\rowcolor{white}
1.0.0 & 2025-10-28 & Gabriele Scaggiante & Stesura iniziale del verbale & Dennis Parolin \\
\hline

\end{tabular}
\end{center}

\end{document}
