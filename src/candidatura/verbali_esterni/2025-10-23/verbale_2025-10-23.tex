% template presente
\documentclass[a4paper,12pt]{article}

\usepackage[utf8]{inputenc}
\usepackage[italian]{babel}
\usepackage{graphicx}
\graphicspath{{resources/}{../resources/}{../../resources/}{../../../resources/}{../../../../resources/}}
\usepackage{xcolor}
\usepackage{geometry}
\usepackage{setspace}
\usepackage{colortbl}
\usepackage{hyperref} % link cliccabili
\usepackage{fancyhdr} % intestazioni personalizzate
\usepackage{titlesec} % per formattare le sezioni
\geometry{margin=2.5cm}

\setlength{\parindent}{0pt}
\setstretch{1.2}

% ===== Stile intestazione =====
\pagestyle{fancy}
\fancyhf{}
\fancyhead[L]{\textcolor{gray}{Verbale Riunione - Gruppo 17}}
\fancyfoot[C]{\thepage}

% ===== Formattazione sezioni =====
\titleformat{\section}{\Large\bfseries}{\thesection.}{0.5em}{}
\titleformat{\subsection}{\large\bfseries}{\thesubsection.}{0.5em}{}

\begin{document}

% ======= HEADER UNIVERSITÀ E GRUPPO CENTRATI VERTICALMENTE =======
\vspace*{\fill}

\begin{center}
    \begin{minipage}{0.25\textwidth}
        \centering
        \includegraphics[width=\linewidth]{logoUni.png}
    \end{minipage}
    \hfill
    \begin{minipage}{0.7\textwidth}
        \raggedright
        {\color{red}\LARGE \textbf{Università degli Studi di Padova}}\\[0.3cm]
        {\large
        Laurea: Informatica\\
        Corso: Ingegneria del Software\\
        Anno Accademico: 2025/26
        }
    \end{minipage}
\end{center}

\vspace{1cm}

\begin{center}
    \begin{minipage}{0.25\textwidth}
        \centering
        \includegraphics[width=\linewidth]{logo.png}
    \end{minipage}
    \hfill
    \begin{minipage}{0.7\textwidth}
        \raggedright
        {\LARGE \textbf{Gruppo 17}}\\[0.3cm]
        {\large
        Nome: BitByBit\\
        Email: swe.bitbybit@gmail.com
        }
    \end{minipage}
\end{center}

\vspace{1.5cm}

\begin{center}
    {\LARGE \textbf{Verbale Riunione Numero 1}}\\[0.2cm]
    {\large \textit{Riepilogo chiamata conoscitiva con Zucchetti}}
\end{center}

\vspace*{\fill}

\clearpage

% ======= INFO GENERALI =======
{\large \textbf{Informazioni generali}}
{\footnotesize
\setstretch{1.1}

\begin{itemize}
    \item \textbf{Data:} 23 ottobre 2025
    \item \textbf{Ora inizio:} 15:30
    \item \textbf{Ora fine:} 16:00
    \item \textbf{Tipo riunione:} Esterna
    \item \textbf{Luogo:} Google Meet
    \item \textbf{Durata:} 30 min
    \item \textbf{Responsabile:} Gabriele Scaggiante
\end{itemize}

\vspace{0.2cm}

\textbf{Partecipanti:}
\begin{itemize}
    \item Gregorio Piccoli (Zucchetti)
    \item Membri del gruppo BitByBit:
    \begin{itemize}
        \item Gabriele Scaggiante
        \item Giovanni Visentin
        \item Dennis Parolin
        \item Ferdinando Fracasso
    \end{itemize}
\end{itemize}

\textbf{Assenti:}
\begin{itemize}
    \item Riccardo Manisi
    \item Marco Sanguin
\end{itemize}
}

\vspace{0.8cm}

% ======= INDICE =======
\clearpage
\tableofcontents
\thispagestyle{empty}
\clearpage

% ---------- ORDINE DEL GIORNO ----------
\section{Ordine del Giorno}
\begin{itemize}
    \item Presentazione reciproca tra il gruppo BitByBit e Zucchetti.
    \item Chiarimenti sul progetto “Second Brain”.
\end{itemize}

% ---------- DISCUSSIONI ----------
\section{Discussioni}

Durante la chiamata conoscitiva con \textbf{Gregorio Piccoli}, referente per il progetto “\textit{Second Brain}” di Zucchetti, si sono affrontati diversi temi tecnici e organizzativi.

\subsection{Implementazione dei 6 cappelli per pensare}
La discussione è cominciata con una domanda sui “6 cappelli per pensare” e su come andrebbero implementati. La risposta è stata che abbiamo completa libertà di implementazione. Possiamo quindi prevedere di esaminare il testo usando contemporaneamente tutti i 6 cappelli, evidenziando poi aree di miglioramento in base alle preferenze dell’utente, oppure gestire la cosa di testa nostra senza grossi problemi.

\subsection{Utilizzo dell'AI e infrastruttura}
Zucchetti dispone di tre macchine con GPU di fascia media, utilizzate per l’elaborazione dei prompt.  
I server ospitano localmente vari modelli LLM (tra 7B e 27B) gestiti tramite \textbf{Ollama}, un framework che permette l’accesso ai modelli via \textbf{API endpoint}.  
Alcuni modelli sono open source e locali (senza costi per token), mentre altri, come \textbf{Gemini}, comportano un costo per token e vengono serviti tramite proxy aziendale.  

L’utilizzo dei modelli è consentito fino al raggiungimento di una soglia di spesa massima non specificata; superato tale limite sarà necessario richiedere la riattivazione del servizio.

\subsection{Interazione con gli LLM}
L’interazione con i modelli avverrà tramite un’API che consente di:
\begin{itemize}
    \item Selezionare il modello desiderato;
    \item Fornire due prompt distinti: uno di \textbf{sistema} (per definire lo stile o il ruolo dell’AI) e uno \textbf{utente} (contenente il testo da elaborare).
\end{itemize}

Non è previsto il training o la progettazione di modelli personalizzati. La possibilità di sviluppare modelli non parametrici non è stata esclusa, ma non è un requisito del progetto.

\subsection{Gestione utenti e architettura applicativa}
È stato consigliato di implementare un sistema di \textbf{login} e un backend di gestione utenti e note. Tuttavia, la gestione dei dati dovrà avvenire in locale per un singolo utente, mantenendo il focus sull’aspetto personale dell’applicazione.  

È stato sconsigliato implementare funzionalità complesse come un grafo di link in stile \textit{Obsidian}, privilegiando invece quelle legate all’utilizzo dell’AI.

\subsection{Funzionalità opzionali e revisioni}
Le funzionalità opzionali possono essere definite liberamente nel corso dello sviluppo. È comunque gradito un confronto iniziale con il referente.  
Non è previsto un calendario fisso di revisioni: eventuali incontri possono essere organizzati in base alle necessità, anche in presenza (la sede Zucchetti è nei pressi di Padova).

\subsection{Interfaccia grafica e altre idee}
L’applicazione dovrà essere \textbf{ottimizzata per desktop in orizzontale}; non è necessario un design verticale per mobile.  

Tra le funzionalità opzionali considerate interessanti:
\begin{itemize}
    \item Analisi di video tramite AI e segmentazione basata sui 6 cappelli;
    \item Digitazione vocale;
    \item Generazione di contenuti testuali a partire da input non testuali.
\end{itemize}

% ---------- DECISIONI ----------
\section{Decisioni}
\begin{itemize}
    \item N.A
\end{itemize}

% ---------- TO DO ----------
\section{To Do}

\begin{center}
\small
\renewcommand{\arraystretch}{1.2} 
\arrayrulecolor{black} 
\begin{tabular}{|p{0.45\linewidth}|p{0.25\linewidth}|p{0.25\linewidth}|}
\hline
\rowcolor{gray!60} 
\textbf{Task} & \textbf{Assegnatario} & \textbf{Scadenza} \\
\hline
\rowcolor{white}
N.A &  &  \\
\hline
\rowcolor{gray!20}
N.A &  &  \\
\hline
\rowcolor{white}
N.A &  &  \\
\hline
\rowcolor{gray!20}
N.A &  &  \\
\hline
\end{tabular}
\end{center}

% ---------- REDAZIONE E REVISIONE ----------
\clearpage
\section{Redazione e revisioni del documento}

\begin{center}
\small
\renewcommand{\arraystretch}{1.2} 
\arrayrulecolor{black}
\begin{tabular}{|p{0.1\linewidth}|p{0.18\linewidth}|p{0.22\linewidth}|p{0.20\linewidth}|p{0.22\linewidth}|}
\hline
\rowcolor{gray!60} 
\textbf{Versione} & \textbf{Ruolo} & \textbf{Nome} & \textbf{Data e ora} & \textbf{Descrizione} \\
\hline
\rowcolor{white}
1.0.0 & Redatto da & Dennis Parolin & 2025-10-23 & Stesura iniziale del verbale \\
\hline
\rowcolor{gray!20}
1.0.0 & Revisione & Gabriele Scaggiante & 2025-10-23 & Controllo approfondito del verbale \\
\hline
\rowcolor{white}
1.0.0 & Conferma & Tutti i membri & 2025-10-28 & Conferma da parte di tutti del verbale \\
\hline
\end{tabular}
\end{center}

% template presente
% ---------- FIRMA AZIENDALE ----------
\vspace*{\fill} % Spinge la firma in fondo alla pagina
\noindent
\begin{minipage}{0.60\textwidth}
    {\small
    \textbf{Firma aziendale:}\\[0.3cm]
    \textit{(Spazio riservato all’azienda per apporre firma o timbro)}\\[0.8cm]
    \fbox{\rule{0pt}{2.5cm}\rule{5cm}{0pt}} % rettangolo segnaposto firma
    }
\end{minipage}
\end{document}
