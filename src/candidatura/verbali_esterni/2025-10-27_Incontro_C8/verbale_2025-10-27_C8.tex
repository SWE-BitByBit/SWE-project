% template presente
\documentclass[a4paper,12pt]{article}

\usepackage[utf8]{inputenc}
\usepackage[italian]{babel}
\usepackage{graphicx}
\graphicspath{{resources/}{../resources/}{../../resources/}{../../../resources/}{../../../../resources/}}
\usepackage{xcolor}
\usepackage{geometry}
\usepackage{setspace}
\usepackage{colortbl}
\usepackage{hyperref} % link cliccabili
\usepackage{fancyhdr} % intestazioni personalizzate
\usepackage{titlesec} % per formattare le sezioni
\geometry{margin=2.5cm}

\setlength{\parindent}{0pt}
\setstretch{1.2}

% ===== Stile intestazione =====
\pagestyle{fancy}
\fancyhf{}
\fancyhead[L]{\textcolor{gray}{Verbale Riunione - Gruppo 17}}
\fancyfoot[C]{\thepage}

% ===== Formattazione sezioni =====
\titleformat{\section}{\Large\bfseries}{\thesection.}{0.5em}{}
\titleformat{\subsection}{\large\bfseries}{\thesubsection.}{0.5em}{}

\begin{document}

% ======= HEADER UNIVERSITÀ E GRUPPO CENTRATI VERTICALMENTE =======
\vspace*{\fill} % --- Spinge verso il basso l'inizio del contenuto

\begin{center}
    \begin{minipage}{0.25\textwidth}
        \centering
        \includegraphics[width=\linewidth]{logoUni.png}
    \end{minipage}
    \hfill
    \begin{minipage}{0.7\textwidth}
        \raggedright
        {\color{red}\LARGE \textbf{Università degli Studi di Padova}}\\[0.3cm]
        {\large
        Laurea: Informatica\\
        Corso: Ingegneria del Software\\
        Anno Accademico: 2025/26
        }
    \end{minipage}
\end{center}

\vspace{1cm}

\begin{center}
    \begin{minipage}{0.25\textwidth}
        \centering
        \includegraphics[width=\linewidth]{logo.png}
    \end{minipage}
    \hfill
    \begin{minipage}{0.7\textwidth}
        \raggedright
        {\LARGE \textbf{Gruppo 17}}\\[0.3cm]
        {\large
        Nome: BitByBit\\
        Email: swe.bitbybit@gmail.com
        }
    \end{minipage}
\end{center}

\vspace{1.5cm}

\begin{center}
    {\LARGE \textbf{Verbale Riunione Numero 2}}
\end{center}

\vspace*{\fill} % --- Spinge verso l’alto la fine del contenuto, centrando tutto il blocco

\clearpage
% ======= INFO GENERALI =======
{\large \textbf{Informazioni generali}}
{\footnotesize
\setstretch{1.1}

\begin{itemize}
    \item \textbf{Data:} 2025-10-27
    \item \textbf{Ora inizio:} 14:00
    \item \textbf{Ora fine:} 14:27
    \item \textbf{Tipo riunione:} Esterna
    \item \textbf{Luogo:} Google Meet
    \item \textbf{Durata:} 27 min
    \item \textbf{Responsabile:} Gabriele Scaggiante
\end{itemize}

\vspace{0.2cm}

\textbf{Partecipanti:}
\begin{itemize}
    \item Gianluca Carlesso (Ergon)
    \item Membri del gruppo BitByBit:
    \begin{itemize}
        \item Gabriele Scaggiante
        \item Giovanni Visentin
        \item Ferdinando Fracasso
        \item Riccardo Manisi
    \end{itemize}
\end{itemize}

\textbf{Assenti:}
\begin{itemize}
    \item Dennis Parolin
    \item Marco Sanguin
\end{itemize}
}

\vspace{0.5cm}

\vspace{0.8cm}

% ======= INDICE SU PAGINA DEDICATA =======
\clearpage
\tableofcontents
\thispagestyle{empty} % senza numero di pagina per l'indice
\clearpage

% ---------- ORDINE DEL GIORNO ----------
\section{Ordine del Giorno}
\begin{itemize}
    \item Presentazione reciproca tra il gruppo BitByBit e Ergon
    \item Chiarimenti e domande sul progetto "SmartOrder"
\end{itemize}

% ---------- DISCUSSIONI ----------
\section{Discussioni}
Durante la chiamata conoscitiva con Gianluca Carlesso, referente per il progetto “SmartOrder” di Ergon, si sono affrontati diversi temi tecnici e organizzativi.

\subsection{Adattabilità del software e gestione del catalogo}
È stato chiarito che il software deve essere in grado di adattarsi a qualsiasi catalogo di prodotti.  
Ogni azienda dispone di un proprio database contenente l’elenco dei prodotti e i relativi dati (nome, codice, prezzo, immagini, ecc.), con una dimensione media di circa 5.000 articoli.

\subsection{Accesso alle API e utilizzo dei modelli linguistici}
Il referente ha confermato che saranno fornite al gruppo le chiavi per l’accesso alle API degli LLM scelti.  
Sono presenti limiti di utilizzo in termini di token, ma negli anni precedenti tali limiti non hanno rappresentato un problema.

\subsection{Tipologie di input e modalità di inserimento degli ordini}
L’utente potrà inserire gli ordini tramite:
\begin{itemize}
    \item chat testuale integrata nell’applicazione;
    \item messaggi vocali, eventualmente collegati ad app di messaggistica esterne;
    \item immagini contenenti ordini scritti a mano (foto di fogli);
    \item opzionalmente foto di prodotti, barcode o QR code.
\end{itemize}
Ogni ordine corrisponde tipicamente a un singolo tipo di input, ma è possibile prevedere l’unione o la frammentazione di ordini in base alla modalità di inserimento.

\subsection{Gestione degli errori e funzionalità di annullamento}
Nel caso in cui l’ordine non venga compreso, l’applicazione potrà:
\begin{itemize}
    \item mostrare un messaggio di errore e richiedere un nuovo input;
    \item proporre un elenco di prodotti che meglio corrispondono alla richiesta.
\end{itemize}
È inoltre prevista la possibilità di cancellare un ordine in corso o già effettuato.

\subsection{Sistema di login e profilo utente}
È necessario implementare un sistema di autenticazione che consenta all’utente di:
\begin{itemize}
    \item accedere con un proprio profilo;
    \item visualizzare e gestire i propri ordini;
    \item annullare o modificare ordini precedenti.
\end{itemize}

\subsection{Riconoscimento multimodale e approccio incrementale}
Si è concordato di iniziare lo sviluppo dal riconoscimento testuale, per poi integrare successivamente l’elaborazione audio e immagini.  
Le componenti vocali e visive dovranno generare testo, così da rientrare nel flusso di interpretazione già previsto per l’input scritto.

\subsection{Utilizzo degli LLM e strategie di adattamento ai cataloghi}
Sono state discusse due principali strategie per adattare il modello linguistico ai cataloghi aziendali:
\begin{enumerate}
    \item \textbf{Fine tuning} sul catalogo, per rendere l’AI più mirata e coerente con i prodotti disponibili;
    \item \textbf{Prompt dinamico} contenente il catalogo in formato testuale, con il limite della lunghezza del prompt.
\end{enumerate}
Entrambe le soluzioni permettono di gestire una base dati soggetta a modifiche nel tempo.

\subsection{Interfaccia grafica e vincoli di gestione}
L’interfaccia dovrà consentire all’utente di:
\begin{itemize}
    \item effettuare ordini;
    \item visualizzarne i contenuti e modificarli;
    \item revocarli in caso di necessità;
    \item consultare il catalogo prodotti, anche in caso di errore o su richiesta.
\end{itemize}
Non è invece previsto che l’utente possa modificare direttamente il database del catalogo, la cui gestione rimane competenza del backend.

% ---------- DECISIONI ----------
\section{Decisioni}
\begin{itemize}
    \item N.A.
\end{itemize}

% ---------- TO DO ----------
\section{To Do}
Dalle discussioni e decisioni intraprese non sono sorte task immediate.

% ---------- REDAZIONE E REVISIONE ----------
\clearpage
\section{Redazione e revisioni del documento}

\begin{center}
\small
\renewcommand{\arraystretch}{1.2} 
\arrayrulecolor{black}
\begin{tabular}{|p{0.1\linewidth}|p{0.18\linewidth}|p{0.22\linewidth}|p{0.20\linewidth}|p{0.22\linewidth}|}
\hline
\rowcolor{gray!60} 
\textbf{Versione} & \textbf{Ruolo} & \textbf{Nome} & \textbf{Data e ora} & \textbf{Descrizione} \\
\hline
\rowcolor{white}
1.0.0 & Redatto da & Gabriele Scaggiante & 2025-10-27 & Stesura iniziale del verbale \\
\hline
\rowcolor{gray!20}
1.0.0 & Revisione & Giovanni Visentin & 2025-10-28 & Controllo approfondito del verbale \\
\hline
\rowcolor{white}
1.0.0 & Conferma & Tutti i membri & 2025-10-29 & Conferma da parte di tutti del verbale \\
\hline

\end{tabular}
\end{center}

% template presente
% ---------- FIRMA AZIENDALE ----------
\vspace*{\fill} % Spinge la firma in fondo alla pagina
\noindent
\begin{minipage}{0.60\textwidth}
    {\small
    \textbf{Firma aziendale:}\\[0.3cm]
    \textit{(Spazio riservato all’azienda per apporre firma o timbro)}\\[0.8cm]
    \fbox{\rule{0pt}{2.5cm}\rule{5cm}{0pt}} % rettangolo segnaposto firma
    }
\end{minipage}

\end{document}