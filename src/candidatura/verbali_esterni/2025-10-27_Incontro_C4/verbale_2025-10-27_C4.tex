% template presente
\documentclass[a4paper,12pt]{article}

\usepackage[utf8]{inputenc}
\usepackage[italian]{babel}
\usepackage{graphicx}
\graphicspath{{resources/}{../resources/}{../../resources/}{../../../resources/}{../../../../resources/}}
\usepackage{xcolor}
\usepackage{geometry}
\usepackage{setspace}
\usepackage{colortbl}
\usepackage{hyperref} % link cliccabili
\usepackage{fancyhdr} % intestazioni personalizzate
\usepackage{titlesec} % per formattare le sezioni
\geometry{margin=2.5cm}

\setlength{\parindent}{0pt}
\setstretch{1.2}

% ===== Stile intestazione =====
\pagestyle{fancy}
\fancyhf{}
\fancyhead[L]{\textcolor{gray}{Verbale Riunione - Gruppo 17}}
\fancyfoot[C]{\thepage}

% ===== Formattazione sezioni =====
\titleformat{\section}{\Large\bfseries}{\thesection.}{0.5em}{}
\titleformat{\subsection}{\large\bfseries}{\thesubsection.}{0.5em}{}

\begin{document}

% ======= HEADER UNIVERSITÀ E GRUPPO CENTRATI VERTICALMENTE =======
\vspace*{\fill} % --- Spinge verso il basso l'inizio del contenuto

\begin{center}
    \begin{minipage}{0.25\textwidth}
        \centering
        \includegraphics[width=\linewidth]{logoUni.png}
    \end{minipage}
    \hfill
    \begin{minipage}{0.7\textwidth}
        \raggedright
        {\color{red}\LARGE \textbf{Università degli Studi di Padova}}\\[0.3cm]
        {\large
        Laurea: Informatica\\
        Corso: Ingegneria del Software\\
        Anno Accademico: 2025/26
        }
    \end{minipage}
\end{center}

\vspace{1cm}

\begin{center}
    \begin{minipage}{0.25\textwidth}
        \centering
        \includegraphics[width=\linewidth]{logo.png}
    \end{minipage}
    \hfill
    \begin{minipage}{0.7\textwidth}
        \raggedright
        {\LARGE \textbf{Gruppo 17}}\\[0.3cm]
        {\large
        Nome: BitByBit\\
        Email: swe.bitbybit@gmail.com
        }
    \end{minipage}
\end{center}

\vspace{1.5cm}

\begin{center}
    {\LARGE \textbf{Verbale Riunione Numero 3}}
\end{center}

\vspace*{\fill} % --- Spinge verso l’alto la fine del contenuto, centrando tutto il blocco

\clearpage
% ======= INFO GENERALI =======
{\large \textbf{Informazioni generali}}
{\footnotesize
\setstretch{1.1}

\begin{itemize}
    \item \textbf{Data:} 2025-10-27
    \item \textbf{Ora inizio:} 16:15
    \item \textbf{Ora fine:} 16:56
    \item \textbf{Tipo riunione:} Esterna
    \item \textbf{Luogo:} Google Meet
    \item \textbf{Durata:} 41 min
    \item \textbf{Responsabile:} Gabriele Scaggiante
\end{itemize}

\vspace{0.2cm}

\textbf{Partecipanti:}
\begin{itemize}
    \item Emanuele Righetto (Miriade)
    \item Arianna Bellino (Miriade)
    \item Membri del gruppo BitByBit:
    \begin{itemize}
        \item Gabriele Scaggiante
        \item Giovanni Visentin
        \item Dennis Parolin 
        \item Riccardo Manisi
    \end{itemize}
\end{itemize}

\textbf{Assenti:}
\begin{itemize}
    \item Ferdinando Fracasso
    \item Marco Sanguin
\end{itemize}
}

\vspace{0.5cm}

\vspace{0.8cm}

% ======= INDICE SU PAGINA DEDICATA =======
\clearpage
\tableofcontents
\thispagestyle{empty} % senza numero di pagina per l'indice
\clearpage

% ---------- ORDINE DEL GIORNO ----------
\section{Ordine del Giorno}
\begin{itemize}
    \item Presentazione reciproca tra il gruppo BitByBit e Miriade
    \item Chiarimenti e domande sul progetto "L'app Che Protegge e Trasforma"
\end{itemize}

% ---------- DISCUSSIONI ----------
\section{Discussioni}
Durante la chiamata conoscitiva con Emanuele Righetto e Arianna Bellino, referenti per il progetto “L’app che Protegge e Trasforma” di Miriade, sono stati affrontati diversi temi relativi alla progettazione, alla gestione dei dati e alla sicurezza dell’applicazione.

\subsection{Gestione dei dati e tutela della privacy}
È stato chiarito che tutte le interazioni tra utente e applicazione, in linea teorica, dovrebbero essere tracciate e memorizzate nella banca dati.  
Esempi includono le domande poste dall’utente all’intelligenza artificiale e le relative risposte. Tuttavia, è necessario prestare particolare attenzione al trattamento dei dati personali: in linea di massima sarebbe meglio non raccogliere dati riconducibili a un singolo individuo, in modo da tutelare la privacy.  
Le informazioni registrate verranno utilizzate per migliorare le prestazioni dell’app, ad esempio fornendo risposte più rapide a richieste già effettuate.

\subsection{Ambito di distribuzione dell’applicazione}
Alla domanda relativa alla diffusione dell’app oltre il territorio nazionale, è stato specificato che, almeno nella fase iniziale, lo sviluppo e la distribuzione avverranno esclusivamente in Italia, al fine di rispettare la normativa vigente e semplificare la gestione legale.

\subsection{Tracciamento dei percorsi sicuri e progettazione funzionale}
Il gruppo ha chiesto chiarimenti sulla gestione dei percorsi sicuri. Non è stata fornita una risposta definitiva sulla necessità di memorizzarli in un database o recuperarli da fonti esterne.  
È stato comunque sottolineato che non tutte le funzionalità richieste sono vincolanti: il gruppo è invitato a proporre soluzioni personali sia dal punto di vista tecnico che di design, contribuendo attivamente alla progettazione funzionale dell’app.

\subsection{Rilevamento passivo di situazioni di pericolo}
Riguardo al meccanismo di rilevamento delle “red flag”, i referenti hanno lasciato libertà di scelta sulle modalità di implementazione, raccomandando di prestare attenzione ai consumi energetici dell'app.  
È stato infatti osservato che mantenere l’app in ascolto continuo potrebbe esaurire rapidamente la batteria. Tra le possibili alternative è stata menzionata l’integrazione con assistenti vocali come Alexa per il rilevamento di episodi potenzialmente pericolosi.

\subsection{Funzionalità di chiamata ai soccorsi}
È stato chiarito che la chiamata ai soccorsi non dovrà essere inviata direttamente ai numeri di emergenza, ma potrà essere sostituita da un sistema di segnalazione, ad esempio tramite:
\begin{itemize}
    \item invio di un segnale broadcast ai dispositivi nelle vicinanze;
    \item notifica a una lista di contatti fidati che potranno decidere se allertare i soccorsi.
\end{itemize}
L’intelligenza artificiale non deve mai avere l’ultima parola: la decisione finale deve sempre spettare a una persona reale.

\subsection{Gestione di assenza di connessione e campo}
È stato affrontato il problema dell’utilizzo dell’app in condizioni di assenza di rete o di segnale telefonico.  
In tali situazioni l’app risulterebbe in gran parte inutilizzabile; pertanto, il design dovrà tenere conto dello stato del dispositivo e dell’ambiente in cui si trova l’utente.

\subsection{Disinstallazione e sicurezza dell’app}
Si è discusso sulla possibilità di distinguere tra una disinstallazione normale e una forzata.  
È stata proposta l’introduzione di una password necessaria per la rimozione dell’app, così da impedire che una persona esterna possa disinstallarla senza autorizzazione.  
È stato inoltre suggerito di camuffare l’applicazione, ad esempio con schermate o nomi ingannevoli, per evitare sospetti in caso di controllo del dispositivo da parte di terzi.

\subsection{Collaborazioni e supporti esterni}
In relazione alle funzionalità con implicazioni legislative o psicologiche, è stata confermata la possibilità di essere messi in contatto con associazioni partner di Miriade, che potranno fornire supporto teorico per elementi come quiz informativi o note normative.

\subsection{Distribuzione e testing dell’app}
È stato chiarito che la pubblicazione dell’app sugli store ufficiali non è a carico del gruppo.  
Sarà sufficiente produrre build di test per piattaforme come TestFlight (iOS) o Android, così da consentire prove su dispositivi reali.

\subsection{Utilizzo dell’intelligenza artificiale}
L’azienda mette a disposizione le API di \textbf{Amazon Bedrock} per l’integrazione dell’intelligenza artificiale.  
I costi relativi all’uso dei token non rappresentano una criticità.

% ---------- DECISIONI ----------
\section{Decisioni}
\begin{itemize}
    \item N.A.
\end{itemize}

% ---------- TO DO ----------
\section{To Do}
Dalle discussioni e decisioni intraprese non sono sorte task immediate.

% ---------- REDAZIONE E REVISIONE ----------
\clearpage
\section{Redazione e revisioni del documento}

\begin{center}
\small
\renewcommand{\arraystretch}{1.2} 
\arrayrulecolor{black}
\begin{tabular}{|p{0.1\linewidth}|p{0.18\linewidth}|p{0.22\linewidth}|p{0.20\linewidth}|p{0.22\linewidth}|}
\hline
\rowcolor{gray!60} 
\textbf{Versione} & \textbf{Ruolo} & \textbf{Nome} & \textbf{Data e ora} & \textbf{Descrizione} \\
\hline
\rowcolor{white}
1.0.0 & Redatto da & Gabriele Scaggiante & 2025-10-27 & Stesura iniziale del verbale \\
\hline
\rowcolor{gray!20}
1.0.0 & Revisione & Giovanni Visentin & 2025-10-28 & Controllo approfondito del verbale \\
\hline
\rowcolor{white}
1.0.0 & Conferma & Tutti i membri & 2025-10-29 & Conferma da parte di tutti del verbale \\
\hline

\end{tabular}
\end{center}

% template presente
% ---------- FIRMA AZIENDALE ----------
\vspace*{\fill} % Spinge la firma in fondo alla pagina
\noindent
\begin{minipage}{0.60\textwidth}
    {\small
    \textbf{Firma aziendale:}\\[0.3cm]
    \textit{(Spazio riservato all’azienda per apporre firma o timbro)}\\[0.8cm]
    \fbox{\rule{0pt}{2.5cm}\rule{5cm}{0pt}} % rettangolo segnaposto firma
    }
\end{minipage}

\end{document}