% template presente
\documentclass[a4paper,12pt]{article}

\usepackage{tabularx}
\usepackage{tcolorbox}
\usepackage[utf8]{inputenc}
\usepackage[italian]{babel}
\usepackage{hyperref}
\hypersetup{
    colorlinks=true,
    linkcolor=blue,
    filecolor=blue,
    urlcolor=blue,     
}
\usepackage{graphicx}
\graphicspath{{resources/}{../resources/}{../../resources/}{../../../resources/}{../../../../resources/}}
\usepackage{xcolor}
\usepackage{geometry}
\usepackage{setspace}
\usepackage{colortbl}
\usepackage{hyperref} 
\usepackage{fancyhdr} 
\usepackage{titlesec}
\geometry{margin=2.5cm}
% ===== Variabile per il titolo del documento =====
\newcommand{\documenttitle}{Verbale Riunione Numero X}


\setlength{\parindent}{0pt}
\setstretch{1.2}

% ===== Stile intestazione =====
\pagestyle{fancy}
\fancyhf{}
\fancyhead[L]{\textcolor{gray}{\documenttitle - BitByBit}}
\fancyfoot[C]{\thepage}

% ===== Formattazione sezioni =====
\titleformat{\section}{\Large\bfseries}{\thesection.}{0.5em}{}
\titleformat{\subsection}{\large\bfseries}{\thesubsection.}{0.5em}{}

\begin{document}

% ======= HEADER UNIVERSITÀ E GRUPPO CENTRATI VERTICALMENTE =======
\vspace*{\fill} % --- Spinge verso il basso l'inizio del contenuto

\begin{center}
    \begin{minipage}{0.25\textwidth}
        \centering
        \includegraphics[width=\linewidth]{logoUni.png}
    \end{minipage}
    \hfill
    \begin{minipage}{0.7\textwidth}
        \raggedright
        {\color{red}\LARGE \textbf{Università degli Studi di Padova}}\\[0.3cm]
        {\large
        Laurea: Informatica\\
        Corso: Ingegneria del Software\\
        Anno Accademico: 2025/26
        }
    \end{minipage}
\end{center}

\vspace{1cm}

\begin{center}
    \begin{minipage}{0.25\textwidth}
        \centering
        \includegraphics[width=\linewidth]{logo.png}
    \end{minipage}
    \hfill
    \begin{minipage}{0.7\textwidth}
        \raggedright
        {\LARGE \textbf{Gruppo 17}}\\[0.3cm]
        {\large
        Nome: BitByBit\\
        Email: swe.bitbybit@gmail.com
        }
    \end{minipage}
\end{center}

\vspace{1.5cm}

\begin{center}
    {\LARGE \textbf{\documenttitle}}
\end{center}

\vspace*{\fill} % --- Spinge verso l’alto la fine del contenuto, centrando tutto il blocco

\clearpage
% ======= INFO GENERALI =======
\section*{Informazioni Generali}
\renewcommand{\arraystretch}{1.3}

\begin{tcolorbox}
\begin{tabularx}{\textwidth}{@{}l X@{}}
\textbf{Redattore:} & XXXX \\
\textbf{Verificatore:} & XXXX \\
\textbf{Responsabile:} & XXXX \\
\textbf{Amministratore:} & XXXX \\
\textbf{Data:} & 23 ottobre 2025 \\
\textbf{Durata:} & 1h 45m \\
\textbf{Luogo:} & Padova, Aula 2 \\
\end{tabularx}
\end{tcolorbox}

\vspace{0.4cm}
\textbf{Partecipanti:}
\begin{itemize}
    \item Manisi Riccardo
    \item Parolin Dennis
    \item Scaggiante Gabriele
    \item Sanguin Marco
    \item Visentin Giovanni
\end{itemize}

\textbf{Assenti:}
\begin{itemize}
    \item (nessuno)
\end{itemize}

\clearpage

\vspace{0.5cm}

\vspace{0.8cm}

% ======= INDICE SU PAGINA DEDICATA =======
\clearpage
\tableofcontents
\thispagestyle{empty} % senza numero di pagina per l'indice
\clearpage

% ---------- ORDINE DEL GIORNO ----------
\section{Ordine del Giorno}
\begin{itemize}
    \item %modifica
\end{itemize}

% ---------- DISCUSSIONI ----------
\section{Discussioni}
%modifica

% ---------- DECISIONI ----------
\section{Decisioni}
\begin{center}
\small
\renewcommand{\arraystretch}{1.2} 
\arrayrulecolor{black} 
\begin{tabular}{|p{0.73\textwidth}|c|}
\hline
\rowcolor{gray!60} 
\textbf{Descrizione decisione} & \textbf{Codice decisione} \\
\hline
\rowcolor{white}
Inserire descrizione & VI C 1.1 \\
\hline
\rowcolor{gray!20}
Inserire descrizione &  VI C 1.2 \\
\hline
\rowcolor{white}
Inserire descrizione &  VI C 1.3\\
\hline
\end{tabular}
\end{center}

% ---------- TO DO ----------
\section{To Do}
Dalle discussioni e decisioni intraprese, sono emerse le seguenti attività:

\begin{center}
\small
\renewcommand{\arraystretch}{1.2} 
\arrayrulecolor{black} 
\begin{tabular}{|p{0.52\textwidth}|c|c|}
\hline
\rowcolor{gray!60} 
\textbf{Task} & \textbf{Codice decisione} & \textbf{N°issue GitHub} \\
\hline
\rowcolor{white}
Esempio Task 1 & VI C 1.1 & \href{https://github.com/<repo>/issues/1}{\textbf{issue}} \\
\hline
\rowcolor{gray!20}
Esempio Task 2 & VI C 1.2 & \href{https://github.com/<repo>/issues/2}{\textbf{issue}} \\
\hline
\rowcolor{white}
Esempio Task 3 & VI C 1.3 & \href{https://github.com/<repo>/issues/3}{\textbf{issue}} \\
\hline
\end{tabular}
\end{center}

% ---------- REDAZIONE E REVISIONE ----------
\clearpage
\section{Redazione e revisioni del documento}

\begin{center}
\small
\renewcommand{\arraystretch}{1.2} 
\arrayrulecolor{black}
\begin{tabular}{|c|p{0.11\linewidth}|p{0.21\linewidth}|p{0.23\linewidth}|p{0.21\linewidth}|}
\hline
\rowcolor{gray!60} 
\textbf{Versione} & \textbf{Data} & \textbf{Autore} & \textbf{Descrizione} & \textbf{Verificatore} \\
\hline
\rowcolor{white}
 &  &  &  &  \\
\hline
\rowcolor{gray!20}
 &  &  &  & \\
\hline
\rowcolor{white}
 &  &  &  & \\
\hline
\rowcolor{gray!20}
 &  &  &  &  \\
\hline
\end{tabular}
\end{center}

\end{document}
