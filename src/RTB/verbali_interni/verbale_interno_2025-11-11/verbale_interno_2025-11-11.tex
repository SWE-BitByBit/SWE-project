% template presente
\documentclass[a4paper,12pt]{article}

\usepackage{tabularx}
\usepackage{tcolorbox}
\usepackage[utf8]{inputenc}
\usepackage[italian]{babel}
\usepackage{hyperref}
\hypersetup{
    colorlinks=true,
    linkcolor=blue,
    filecolor=blue,
    urlcolor=blue,     
}
\usepackage{graphicx}
\graphicspath{{resources/}{../resources/}{../../resources/}{../../../resources/}{../../../../resources/}}
\usepackage{xcolor}
\usepackage{geometry}
\usepackage{setspace}
\usepackage{colortbl}
\usepackage{hyperref} 
\usepackage{fancyhdr} 
\usepackage{titlesec}
\geometry{margin=2.5cm}
% ===== Variabile per il titolo del documento =====
\newcommand{\documenttitle}{Verbale Riunione Numero 1 - RTB}


\setlength{\parindent}{0pt}
\setstretch{1.2}

% ===== Stile intestazione =====
\pagestyle{fancy}
\fancyhf{}
\fancyhead[L]{\textcolor{gray}{\documenttitle - BitByBit}}
\fancyfoot[C]{\thepage}

% ===== Formattazione sezioni =====
\titleformat{\section}{\Large\bfseries}{\thesection.}{0.5em}{}
\titleformat{\subsection}{\large\bfseries}{\thesubsection.}{0.5em}{}

\begin{document}

% ======= HEADER UNIVERSITÀ E GRUPPO CENTRATI VERTICALMENTE =======
\vspace*{\fill} % --- Spinge verso il basso l'inizio del contenuto

\begin{center}
    \begin{minipage}{0.25\textwidth}
        \centering
        \includegraphics[width=\linewidth]{logoUni.png}
    \end{minipage}
    \hfill
    \begin{minipage}{0.7\textwidth}
        \raggedright
        {\color{red}\LARGE \textbf{Università degli Studi di Padova}}\\[0.3cm]
        {\large
        Laurea: Informatica\\
        Corso: Ingegneria del Software\\
        Anno Accademico: 2025/26
        }
    \end{minipage}
\end{center}

\vspace{1cm}

\begin{center}
    \begin{minipage}{0.25\textwidth}
        \centering
        \includegraphics[width=\linewidth]{logo.png}
    \end{minipage}
    \hfill
    \begin{minipage}{0.7\textwidth}
        \raggedright
        {\LARGE \textbf{Gruppo 17}}\\[0.3cm]
        {\large
        Nome: BitByBit\\
        Email: swe.bitbybit@gmail.com
        }
    \end{minipage}
\end{center}

\vspace{1.5cm}

\begin{center}
    {\LARGE \textbf{\documenttitle}}
\end{center}

\vspace*{\fill} % --- Spinge verso l’alto la fine del contenuto, centrando tutto il blocco

\clearpage
% ======= INFO GENERALI =======
\section*{Informazioni Generali}
\renewcommand{\arraystretch}{1.3}

\begin{tcolorbox}
\begin{tabularx}{\textwidth}{@{}l X@{}}
\textbf{Redattore:} & Manisi Riccardo \\
\textbf{Data:} & 11 novembre 2025 \\
\textbf{Durata:} & 1h \\
\textbf{Luogo:} & Discord \\
\end{tabularx}
\end{tcolorbox}

\vspace{0.4cm}
\textbf{Partecipanti:}
\begin{itemize}
    \item Manisi Riccardo
    \item Parolin Dennis
    \item Scaggiante Gabriele
    \item Sanguin Marco
    \item Visentin Giovanni
    \item Ferdinando Fracasso
\end{itemize}

\textbf{Assenti:}
\begin{itemize}
    \item (nessuno)
\end{itemize}

\clearpage

\vspace{0.5cm}

\vspace{0.8cm}

% ======= INDICE SU PAGINA DEDICATA =======
\clearpage
\tableofcontents
\thispagestyle{empty} % senza numero di pagina per l'indice
\clearpage

% ---------- ORDINE DEL GIORNO ----------
\section{Ordine del Giorno}
\begin{itemize}
    \item Scegliere il periodo di sprint da concordare anche con la proponente.
    \item Discussione dell'aggiudicazione del capitolato e stesura bozza per la prima mail da mandare alla poroponente.
    \item Presa visione dei documenti che ci saranno da redigere nel periodo successivo.
    \item Analisi degli argomenti di studio per un buon proseguimento del progetto.
\end{itemize}

% ---------- DISCUSSIONI ----------
\section{Discussioni}
\begin{itemize}
    \item A fronte delle spiegazione avvenute in aula riguardo la gestione dei ruoli è stata redatta una tabella per la spartizione delle ore per persona per ruolo. Questo è stato solo un primo tentativo come base per i prossimi sprint.
    \item Sono stati discusse le criticità presentate dal Prof.Vardanega all'assegnazione dei capitolati ai vari gruppi.
    \item Si è decisa la stesusa di una prima bozza della mail che verrà mandata alla proponente per un'primo contatto conoscitivo.
    \item Proposta per l'inizio dei primi documenti interni quali le norme di progetto e il glossario. Una prima parte del documento deve essere redatta per avere un way of working comune a cui si può fare rifermineto per la redazione dei vari documenti. A fronte dei dubbi si è stabilito che servirà ancora del tempo di studio per avere la completa comprensione delle altre sezioni.
    \item Si è identificato come argomenti di studio comune al gruppo anche il file Piano di progetto per sapere i documenti che si inizieranno a fare dai prossimi sprint.
\end{itemize}

% ---------- DECISIONI ----------
\section{Decisioni}
\begin{center}
\small
\renewcommand{\arraystretch}{1.2} 
\arrayrulecolor{black} 
\begin{tabular}{|p{0.73\textwidth}|c|}
\hline
\rowcolor{gray!60} 
\textbf{Descrizione decisione} & \textbf{Codice decisione} \\
\hline
\rowcolor{white}
Iniziare la stesura del documento interno norme di progetto per insierire il processo secondario di documentazione. & VI RTB 1.1 \\
\hline
\rowcolor{gray!20}
Miglioramento della repository a fronte dei commenti del Prof.Vardanega &  VI RTB 1.2 \\
\hline
\rowcolor{white}
Mandare la mail a Miriade per il primo incontro. Stabilito il periodo che ci andrebbe bene per il primo incontro con la proponente. &  VI RTB 1.3\\
\hline
\rowcolor{gray!20}
Studio preliminare delle sezioni e dei contenuti che devono esserci presenti nel file Piano di progetto &  VI RTB 1.4\\
\hline
\rowcolor{white}
Miglioramento del way of working seguendo i problemi riscontrati lavorando alla repository. &  VI RTB 1.5\\
\hline
\end{tabular}
\end{center}

% ---------- TO DO ----------
\section{To Do}
Dalle discussioni e decisioni intraprese, sono emerse le seguenti attività:

\begin{center}
\small
\renewcommand{\arraystretch}{1.2} 
\arrayrulecolor{black} 
\begin{tabular}{|p{0.52\textwidth}|c|c|}
\hline
\rowcolor{gray!60} 
\textbf{Task} & \textbf{Codice decisione} & \textbf{N°issue GitHub} \\
\hline
\rowcolor{white}
Invio della mail a Miriade. & VI RTB 1.3 & \href{https://github.com/SWE-BitByBit/SWE-project/issues/71}{\textbf{71}} \\
\hline
\rowcolor{gray!20}
Norme di progetto studio per capire come strutturare il documento ed inizio redazione secondo standard. & VI RTB 1.1 & \href{https://github.com/SWE-BitByBit/SWE-project/issues/72}{\textbf{72}} \\
\hline
\rowcolor{white}
Sistemazione del template per verbali interni, problemi riscontrati attraverso l’aggiudicazione. & VI RTB 1.2 & \href{https://github.com/SWE-BitByBit/SWE-project/issues/73}{\textbf{73}} \\
\hline
\rowcolor{gray!20}
Studio come redigere al meglio il piano di progetto. & VI RTB 1.2 & \href{https://github.com/SWE-BitByBit/SWE-project/issues/74}{\textbf{74}} \\
\hline
\rowcolor{white}
Studio way of working & VI RTB 1.5 & \href{https://github.com/SWE-BitByBit/SWE-project/issues/75}{\textbf{75}} \\
\hline
\rowcolor{gray!20}
setup della repo github per seguire il way of working & VI RTB 1.5 & \href{https://github.com/SWE-BitByBit/SWE-project/issues/76}{\textbf{76}} \\
\hline
\end{tabular}
\end{center}

% ---------- REDAZIONE E REVISIONE ----------
\clearpage
\section{Redazione e revisioni del documento}

\begin{center}
\small
\renewcommand{\arraystretch}{1.2} 
\arrayrulecolor{black}
\begin{tabular}{|c|p{0.11\linewidth}|p{0.21\linewidth}|p{0.23\linewidth}|p{0.21\linewidth}|}
\hline
\rowcolor{gray!60} 
\textbf{Versione} & \textbf{Data} & \textbf{Autore} & \textbf{Descrizione} & \textbf{Verificatore} \\
\hline
\rowcolor{white}
v0.1.0 & 2025-15-11 & Riccardo Manisi & Stesura del primo incontro successivo all'aggiudicazione del capitolato &  \\
\hline
\end{tabular}
\end{center}

\end{document}
