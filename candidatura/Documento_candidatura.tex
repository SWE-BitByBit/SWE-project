% template presente
\documentclass[a4paper,12pt]{article}

\usepackage[utf8]{inputenc}
\usepackage[italian]{babel}
\usepackage{graphicx}
\graphicspath{{resources/}{../resources/}{../../resources/}{../../../resources/}{../../../../resources/}}
\usepackage{xcolor}
\usepackage{geometry}
\usepackage{setspace}
\usepackage{colortbl}
\usepackage{hyperref} % link cliccabili
\usepackage{fancyhdr} % intestazioni personalizzate
\usepackage{titlesec} % per formattare le sezioni
\geometry{margin=2.5cm}

\setlength{\parindent}{0pt}
\setstretch{1.2}

% ===== Stile intestazione =====
\pagestyle{fancy}
\fancyhf{}
\fancyhead[L]{\textcolor{gray}{Verbale Riunione - Gruppo 17}}
\fancyfoot[C]{\thepage}

% ===== Formattazione sezioni =====
\titleformat{\section}{\Large\bfseries}{\thesection.}{0.5em}{}
\titleformat{\subsection}{\large\bfseries}{\thesubsection.}{0.5em}{}

\begin{document}

% ======= HEADER UNIVERSITÀ E GRUPPO CENTRATI VERTICALMENTE =======
\vspace*{\fill} % --- Spinge verso il basso l'inizio del contenuto

\begin{center}
    \begin{minipage}{0.25\textwidth}
        \centering
        \includegraphics[width=\linewidth]{logoUni.png}
    \end{minipage}
    \hfill
    \begin{minipage}{0.7\textwidth}
        \raggedright
        {\color{red}\LARGE \textbf{Università degli Studi di Padova}}\\[0.3cm]
        {\large
        Laurea: Informatica\\
        Corso: Ingegneria del Software\\
        Anno Accademico: 2025/26
        }
    \end{minipage}
\end{center}

\vspace{1cm}

\begin{center}
    \begin{minipage}{0.25\textwidth}
        \centering
        \includegraphics[width=\linewidth]{logo.png}
    \end{minipage}
    \hfill
    \begin{minipage}{0.7\textwidth}
        \raggedright
        {\LARGE \textbf{Gruppo 17}}\\[0.3cm]
        {\large
        Nome: BitByBit\\
        Email: swe.bitbybit@gmail.com
        }
    \end{minipage}
\end{center}

\vspace{1.5cm}

\begin{center}
    {\LARGE \textbf{Documento Candidatura}}
\end{center}

\vspace*{\fill} % --- Spinge verso l’alto la fine del contenuto, centrando tutto il blocco

\clearpage

% ======= INDICE SU PAGINA DEDICATA =======
\clearpage
\tableofcontents
\thispagestyle{empty} % senza numero di pagina per l'indice
\clearpage

% ---------- TO DO ----------
\section{Candidatura}

Il gruppo \textbf{BitByBit} intende ufficialmente candidarsi per lo sviluppo del \textbf{ !!! }, proposto dall’azienda \textbf{ !!! }

La decisione è stata presa a seguito di un’analisi approfondita dei capitolati proposti, valutando in particolare:
\begin{itemize}
    \item l’interesse tecnico e formativo del progetto;
    \item la fattibilità rispetto alle competenze del gruppo.
\end{itemize}

\subsection{Direttive interne e organizzazione della repository}

Al fine di mantenere ordine e tracciabilità, il gruppo stabilisce le seguenti linee guida operative per la visione della repository:
\begin{itemize}
    \item Tutti i documenti ufficiali sono contenuti nella cartella \texttt{/Documenti/};
    \item I verbali interni sono raccolti in: \texttt{/Documenti/VerbaliInterni/}
    \item I verbali esterni con le aziende:
    \begin{enumerate}
        \item \textbf{Zucchetti S.p.A.} – per informazioni sul capitolato C6 \emph{Second Brain};
        \item \textbf{Ergon} – per chiarimenti sul capitolato C8 \emph{Smart Order};
        \item \textbf{Miriade} – per il capitolato C4 \emph{L'app che protegge e trasforma}.
    \end{enumerate}
    e sono raccolti in  \texttt{/Documenti/VerbaliEsterni/}
    
    \item L’\textbf{Analisi dei Capitolati} è disponibile in \texttt{/Documenti/AnalisiCapitolati.pdf};
    \item Il \textbf{Preventivo di spese e impegni} si trova in \texttt{/Documenti/Preventivo.pdf};
\end{itemize}

\subsection{Composizione del gruppo}

\begin{center}
\small
\renewcommand{\arraystretch}{1.2}
\begin{tabular}{|p{0.3\linewidth}|p{0.2\linewidth}|}
\hline
\rowcolor{gray!60}
\textbf{Nome e Cognome} & \textbf{Matricola} \\
\hline
Dennis Parolin & 2113203 \\
\hline
Ferdinando Fracasso & 2122649 \\
\hline
Giovanni Visentin & 2101064 \\
\hline
Riccardo Manisi & 2111948 \\
\hline
Marco Sanguin & 2103121 \\
\hline
Gabriele Scaggiante & 2101076 \\
\hline
\end{tabular}
\end{center}

% ---------- REDAZIONE E REVISIONE ----------
\clearpage
\section{Redazione e revisioni del documento}

\begin{center}
\small
\renewcommand{\arraystretch}{1.2} 
\arrayrulecolor{black}
\begin{tabular}{|p{0.1\linewidth}|p{0.18\linewidth}|p{0.22\linewidth}|p{0.20\linewidth}|p{0.22\linewidth}|}
\hline
\rowcolor{gray!60} 
\textbf{Versione} & \textbf{Ruolo} & \textbf{Nome} & \textbf{Data e ora} & \textbf{Descrizione} \\
\hline
\rowcolor{white}
0.1 & Redatto da & Giovanni Visentin & 30/10/25 & Stesura iniziale del verbale \\
\hline
\rowcolor{gray!20}
 & Revisione &  &  & Controllo approfondito del verbale \\
\hline
\rowcolor{white}
 & Conferma & Tutti i membri &  & Conferma da parte di tutti del verbale \\
\hline
\rowcolor{gray!20}
 & Modifiche &  &  &  \\
\hline
\end{tabular}
\end{center}

\end{document}
